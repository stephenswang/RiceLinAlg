

\beginedxvertical{Page One}

\beginedxtext{Preliminaries}


At the end of this sequence, and after some practice, you should be able to:

\begin{itemize}
\item Find the standard matrix for a linear transformation from $\R^n$ to $\R^m$.   
\item Recognize what geometric properties are preserved by linear transformations from 
$\R^n$ to $\R^m$.  
\end{itemize}

For time budgeting purposes, this sequence has 5 videos totaling 31 minutes, 
plus some questions.  

% Remember, when you're doing the online learning sequences, you may seek help if you 
% do not understand a video, but you should think about all of the questions 
% entirely individually.  You have pledged to do so under the Honor Code!  


\endedxtext

\endedxvertical

\beginedxvertical{Introduction}



\doedxvideo{Matrix Multiplication as Linear Transformation}{jMRJ-efZOJs}


\beginedxproblem{Where does it go?}{\dpa1}

Consider the following image.  

\begin{center}
\includesvg[400]{c1s8imageunderlintrans} 
\end{center}

When we apply the linear transformation $T: \R^2 \rightarrow \R^2$ given by $T(x) = \left[ \begin{array}{cc}
-1 & 0 \\ 0 & 1 \end{array} \right]$ to the image, which picture is the result?

\begin{center}
\includesvg[400]{c1s8imageunderlintransAB} 
\\
\includesvg[400]{c1s8imageunderlintransCD} 

\end{center}

\edXabox{expect="D" options="A","B","C","D"}


% When we apply the linear transformation $T: \R^2 \rightarrow \R^2$ given by $T(x) = \left[ \begin{array}{cc}
% -1 & 0 \\ 0 & -1 \end{array} \right]$ to the image, which picture is the result?

% \edXabox{expect="A" options="A","B","C","D"}


\edXsolution{We see that $T\Bigg(\left[ \begin{array}{c}
a\\ b\end{array} \right]\Bigg)=\left[ \begin{array}{cc}
-1 & 0 \\ 0 & 1 \end{array} \right]\left[ \begin{array}{c}
a\\ b\end{array}\right]=\left[ \begin{array}{c}
-a\\ b\end{array}\right]$ and thus $T$ reflects the original image over the $y-$axis.

}

\endedxproblem


\beginedxproblem{Where does it go? 2}{\dpa1}

Consider the following image.  

\begin{center}
\includesvg[400]{c1s8imageunderlintrans} 
\end{center}

When we apply the linear transformation $T: \R^2 \rightarrow \R^2$ given by $T(x) = \left[ \begin{array}{cc}
0.5 & 0 \\ 0 & 2 \end{array} \right]$ to the image, which picture is the result?

\begin{center}
\includesvg[400]{c1s8imageunderlintransAB2} 

\end{center}

\edXabox{expect="B" options="A","B"}


% When we apply the linear transformation $T: \R^2 \rightarrow \R^2$ given by $T(x) = \left[ \begin{array}{cc}
% -1 & 0 \\ 0 & -1 \end{array} \right]$ to the image, which picture is the result?

% \edXabox{expect="A" options="A","B","C","D"}


\edXsolution{We see that \[ T\Bigg(\left[ \begin{array}{c}
a\\ b\end{array} \right]\Bigg)=\left[ \begin{array}{cc}
0.5 & 0 \\ 0 & 2 \end{array} \right]\left[ \begin{array}{c}
a\\ b\end{array}\right]=\left[ \begin{array}{c}
0.5a\\ 2b\end{array}\right],\] and thus scales the original image by a factor of 0.5 in the $x$ direction and a factor of 2 in the $y$ direction.

}

\endedxproblem




\endedxvertical


\beginedxvertical{Multiplication by a Matrix in General}

\doedxvideo{Multiplication by a General Matrix}{R1uBwxvKIXE}




\beginedxproblem{The e Vectors}{\dpa2}

Suppose $T: \R^4\rightarrow \R^3$ is given by 
$T(x) = Ax$ where  
\[ A = \left[ \begin{array}{cccc} 1 & 3 & 5 & 7 \\ 2 & 0 & -1 & -1 \\ 1 & 1 & 3  & -2 \end{array} \right].\]



What is $T(e_3)$?  

\input{vectorentry.tex}

\edXabox{type="custom" cfn="VectorEntry" expect="[[5],[-1],[3]]"}


\edXsolution{ Since $T: \R^4\rightarrow \R^3$, we know from context that $e_3=\left[ \begin{array}{c} 0 & 0 & 1 & 0 \end{array} \right].$  Then \[
T(e_3)=\left[ \begin{array}{cccc} 1 & 3 & 5 & 7 \\ 2 & 0 & -1 & -1 \\ 1 & 1 & 3  & -2 \end{array} \right]\left[ \begin{array}{c} 0 \\ 0 \\ 1 \\ 0 \end{array} \right] = \left[ \begin{array}{c} 5 \\ -1 \\ 3 \end{array} \right].\]

Alternatively, if the four columns of $A$ are $v_1, v_2, v_3, v_4$, multiplying $A$ by $\left[ \begin{array}{c} 0 \\ 0 \\ 1 \\ 0 \end{array} \right]$ should yield the linear combination $0v_1 + 0v_2 + 1v_3 + 0v_4,$ which is just
$v_3$.  
}


\endedxproblem

\beginedxproblem{Find a Matrix}{\dpa5}

Suppose the function $T: \R^3\rightarrow \R^2$ is given by 
\[T\left( \left[ \begin{array}{c} a_1 \\ a_2\\ a_3 \end{array} \right] \right) =\left[ \begin{array}{c} 2a_1-a_3 \\ -a_1 + a_2 \end{array} \right] \] 
for all vectors $\left[ \begin{array}{c} a_1 \\ a_2\\ a_3 \end{array} \right]  \in \R^3$.  

%In the actual program, the directins for how students to enter the matrix includes an augmented matrix.  This seems a bit confusing.

For what matrix $A$ is $T(x) = Ax$ for all $x \in \R^3$?  

\input{matrixentry.tex}

\edXabox{type="custom" cfn="MatrixEntry" expect="[[2,0,-1],[-1,1,0]]"}

Is $T$ a linear transformation?  


\edXabox{expect="Yes" options="Yes","No"}

\edXsolution{ 
We see that
\[T\left[ \begin{array}{c} a_1 \\ a_2 \\a_3 \end{array} \right] = 
\left[ \begin{array}{ccc} 2 & 0 & -1 \\ -1  & 1 & 0 \end{array} \right]
\left[ \begin{array}{c} a_1 \\ a_2 \\ a_3 \end{array} \right], \] so 
$A = \left[ \begin{array}{ccc} 2 & 0 & -1 \\ -1  & 1 & 0 \end{array} \right]$ works.  

Since $T$ is multiplication by a matrix, it is a linear transformation.

}


\endedxproblem


\endedxvertical








\beginedxvertical{Matrix for Rotation}


\beginedxproblem{Rotation}{\dpa3}

For the next two problems, we will be considering the linear transformation $T: \R^2 \rightarrow \R^2$
given by rotation counterclockwise by $45^\circ$.  


\begin{center}
\includesvg[400]{c1s8rotation} 

\end{center}

%Is the first sentence below as you intended? Should it be "$45^\circ, 45^\circ, 90^\circ$ triangles"?

From the $45^\circ-45^\circ-90^\circ$ triangles in the diagram, we can find that  
$T(e_1) = \left[ \begin{array}{c} \frac{\sqrt{2}}{2} \\ \frac{\sqrt{2}}{2}  \end{array} \right] \approx
\left[ \begin{array}{c} 0.707 \\ 0.707 \end{array} \right]$ and
$T(e_2) = \left[ \begin{array}{c} -\frac{\sqrt{2}}{2} \\ \frac{\sqrt{2}}{2}  \end{array} \right] \approx
\left[ \begin{array}{c} -0.707 \\ 0.707  \end{array} \right].$   

What is $T\left( \left[ \begin{array}{c} 4 \\ 0 \end{array} \right] \right)$?  (Remember to use decimals only, to within two decimal places.)  

\input{vectorentry2.tex}

\edXabox{type="custom" cfn="VectorEntry" expect="[[2.83],[2.83]]"}


\edXsolution{ Since $T$ is a linear transformation, \[T\left( \left[ \begin{array}{c} 4 \\ 0 \end{array} \right] \right)=4 T\left( \left[ \begin{array}{c} 1 \\ 0 \end{array} \right] \right)=4 T(e_1)=4\left[ \begin{array}{c} \frac{\sqrt{2}}{2} \\ \frac{\sqrt{2}}{2}  \end{array} \right]= \left[ \begin{array}{c} 2\sqrt{2} \\ 2\sqrt{2}  \end{array} \right]\approx
\left[ \begin{array}{c} 2.83 \\ 2.83 \end{array} \right].\]

}


\endedxproblem

\beginedxproblem{Rotation 2}{\dpa3}


What is $T\left( \left[ \begin{array}{c} 4 \\ 1 \end{array} \right] \right)$?  

\input{vectorentry2.tex}

\edXabox{type="custom" cfn="VectorEntry" expect="[[2.12],[3.54]]"}


\edXsolution{ Since $T$ is a linear transformation, \[T\left( \left[ \begin{array}{c} 4 \\ 1 \end{array} \right] \right)=T(4e_1+e_2)=4T(e_1)+T(e_2)=4\cdot \left[ \begin{array}{c} \frac{\sqrt{2}}{2} \\ \frac{\sqrt{2}}{2}  \end{array} \right]+\left[ \begin{array}{c} -\frac{\sqrt{2}}{2} \\ \frac{\sqrt{2}}{2}  \end{array} \right] \approx
\left[ \begin{array}{c} 2.12 \\ 3.54 \end{array} \right].\]

}


\endedxproblem


\doedxvideo{Matrix for Rotation by 45 Degrees}{TN4eIiSNzcA}

\beginedxproblem{Rotation 3}{\dpa3}


If $T$ is still rotation counterclockwise by $45^\circ$, 
what is $T\left( \left[ \begin{array}{c} 3\sqrt{2} \\ -\sqrt{2} \end{array} \right] \right)$?  

\input{vectorentry2.tex}

\edXabox{type="custom" cfn="VectorEntry" expect="[[4],[2]]"}


\edXsolution{

We could do this using linearity properties, but it is easier to just use the matrix $A = \left[ 
\begin{array}{cc} \frac{\sqrt{2}}{2} & -\frac{\sqrt{2}}{2} \\ \frac{\sqrt{2}}{2} & \frac{\sqrt{2}}{2} \end{array} \right]$ we found in the
video.  We know that $T(x) = Ax$ for all $x \in \R^2$, so
\[ T\left( \left[ \begin{array}{c} 3\sqrt{2} \\ -\sqrt{2} \end{array} \right] \right) = 
\left[ 
\begin{array}{cc} \frac{\sqrt{2}}{2} & -\frac{\sqrt{2}}{2} \\ \frac{\sqrt{2}}{2} & \frac{\sqrt{2}}{2} \end{array} \right] \left[ \begin{array}{c} 3\sqrt{2} \\ -\sqrt{2} \end{array} \right] = 
\left[ \begin{array}{c} 4\\2  \end{array} \right]. \]

%  Again, by linearity properties, $T\left( \left[ \begin{array}{c} 3\sqrt{2} \\ -\sqrt{2} \end{array} \right] \right)=3\sqrt{2}\cdot T(e_1)+-\sqrt{2}\cdot T(e_2)=3\sqrt{2}\cdot \left[ \begin{array}{c} \frac{\sqrt{2}}{2} \\ \frac{\sqrt{2}}{2}  \end{array} \right]+-\sqrt{2}\cdot \left[ \begin{array}{c} -\frac{\sqrt{2}}{2} \\ \frac{\sqrt{2}}{2}  \end{array} \right]=\left[ \begin{array}{c} 4\\2  \end{array} \right]$.

}


\endedxproblem


\endedxvertical





\beginedxvertical{Standard Matrix}



\doedxvideo{The Standard Matrix for a Linear Transformation}{aPUQJWsg9Vw}

\beginedxtext{Standard Matrix}

{\keya{\bf{Proposition/Definition.}}} If $T: \R^n \rightarrow \R^m$ is a linear transformation, then there is a unique $m \times n$ matrix
$A$ such that $T(x) = Ax$ for all $x\in \R^n$.  This matrix $A$
is called the {\keyb{\bf standard
matrix}} for $T$, and must be given by the formula
\[A =  \left[ \begin{array}{cccc} | & | & & | \\ 
T(e_1) & T(e_2) & \cdots & T(e_n) \\
 | & | & & | \end{array} \right], \]
 where $e_i \in \R^n$ is the vector whose $i$th entry equals 1, and all of whose other entries are 0.

\endedxtext

\endedxvertical





\beginedxvertical{Standard Matrix Questions}

\beginedxproblem{True or False}{\dpa1}


True or false: Every linear transformation from a vector space $V$ to a vector space $W$
has a standard matrix.  
 
\edXabox{expect="False" options="True","False"}


\edXsolution{ 
False.  Only linear transformations from $\R^n$ to $\R^m$ (or, more generally, from $F^n$ to $F^m$ where
$F$ is a field) have standard matrices.  
}


\endedxproblem


\beginedxproblem{Standard Matrix 1}{\dpa3}

For all $v\in \R^2$, let $T(v)$ be the reflection of $v$ across the horizontal axis of $\R^2$.  
$T$ is a linear transformation from $\R^2$ to $\R^2$.  What is its standard matrix?
 
\input{matrixentry2.tex}

\edXabox{type="custom" cfn="MatrixEntry" expect="[[1,0],[0,-1]]"}


\edXsolution{ If $T$ is a reflection across the horizontal axis, then $e_1$ is fixed by $T$ and thus the first column of the standard matrix is $T(e_1)=e_1$.  Now, $e_2$ is reflected to $-e_2$, and thus the second column of the standard matrix is $T(e_2)=-e_2$.  So the answer is 
$\left[ 
\begin{array}{cc} 1& 0 \\ 0 & -1 \end{array} \right].$
}


\endedxproblem


\beginedxproblem{Standard Matrix 2}{\dpa3}

For all $v\in \R^2$, let $S(v)$ be the vector obtained by reflecting $v$ across the diagonal line $x=y$ 
and 
then rotating the result by $90^\circ$ counterclockwise.  
$S$ is a linear transformation from $\R^2$ to $\R^2$.  What is its standard matrix?
 
\input{matrixentry2.tex}

\edXabox{type="custom" cfn="MatrixEntry" expect="[[-1,0],[0,1]]"}


\edXsolution{ First, under the reflection, $e_1$ is taken to $e_2$.  Then, under the rotation, this vector is taken to $-e_1$.  Thus, $S(e_1)=-e_1$; this will be the first column of the standard matrix.  Similarly, $e_2$ is first taken to $e_1$ and then to $e_2$, and thus $S(e_2)=e_2$; this is the second column.
So the answer is 
$\left[ 
\begin{array}{cc} -1& 0 \\ 0 & 1 \end{array} \right].$
}


\endedxproblem



\beginedxproblem{How many linear transformations?}{\dpa3}

How many linear transformations $T:\R^4 \rightarrow \R^2$ satisfy 
the conditions
$T(e_1) = T(e_3) =  \left[\begin{array}{c} 2 \\ 3   \end{array} \right]$
and $T(e_4) = \veco$?  

\edXabox{type="multichoice" expect="infinitely many" options="zero","one","more than one but finitely many","infinitely many"}

\edXsolution{ This question is equivalent to asking 'how many $2\times 4$ real matrices $A$ satisfy $A e_1=A\cdot e_3=\left[\begin{array}{c} 2 \\ 3   \end{array} \right]$ and $Ae_4=\veco$?'.  The first equality forces the first and third columns of $A$ to both be $\left[\begin{array}{c} 2 \\ 3   \end{array} \right]$, and the second equality forces the fourth column of $A$ to be $\veco$, but the third column of $A$ is free to have any real entries.  
Thus there are infinitely many choices.  
}


\endedxproblem


\beginedxproblem{Identity Transformation}{\dpa1}

One very special linear transformation is the transformation $T: \R^n \rightarrow \R^n$ which is defined
simply by $T(v) = v$ for all $v\in \R^n$.  This is called the identity transformation on $\R^n$, since it takes
every vector to itself.  

What is the standard matrix for the identity transformation on $\R^3$?
 
\input{matrixentry2.tex}

\edXabox{type="custom" cfn="MatrixEntry" expect="[[1,0,0],[0,1,0],[0,0,1]]"}


\edXsolution{ 
We know that the columns of the standard matrix will be $T(e_1),T(e_2),T(e_3)$.  When $T$ is the
identity transformation, this means that the columns will be $e_1, e_2, e_3$.  Hence we get
the matrix $\left[\begin{array}{ccc} 1 & 0 & 0 \\ 0 & 1 & 0 \\ 0 & 0 & 1  \end{array} \right].$ 
}


\endedxproblem

\endedxvertical





\beginedxvertical{The Identity Matrix}



\beginedxtext{Identity Matrix}


{\keya{\bf{Definition.}}} The standard matrix for the identity transformation $T: \R^n \rightarrow \R^n$
is denoted by the symbol $I_n$ and is called the $n\times n$ {\keya{\bf{identity matrix}}}.  Often we simply
use $I$ to denote the identity matrix when the $n$ is understood.  

As in the previous problem, we see that 
\[ I_n = \left[ \begin{array}{cccc} | & | & & | \\ 
e_1 & e_2 & \cdots & e_n \\
 | & | & & | \end{array} \right] = \left[\begin{array}{cccc} 1 & 0 &\cdots & 0 \\ 0 & 1 & \cdots 0 \\ \vdots & \vdots & \ddots & \vdots
\\ 0 & 0 & \cdots & 1  \end{array} \right]. \] 

Since $I_n$ is the standard matrix for the identity transformation $T$, we have 
$I_n v = T(v) = v$ for all $v\in \R^n$.  

\endedxtext



\endedxvertical





\beginedxvertical{Back to Pictures}


\doedxvideo{More Pictures}{8zRN403o8tg}



\beginedxproblem{What is preserved?}{\dpa1}

Take a look at the video again.  Which properties seem to be preserved by linear transformations
from $\R^2$ to $\R^2$?  


For instance, does every transformation preserve the area of regions? 

\edXabox{expect="No" options="Yes","No"}


Do lines that start parallel always go in the same direction after transformation?  


\edXabox{expect="Yes" options="Yes","No"}

Do lines that start perpendicular remain perpendicular after transformation? 

\edXabox{expect="No" options="Yes","No"}

\edXsolution{We see from the video that area is not preserved -- the image can be much smaller or bigger
after transformation than the original.  Similarly, at time 2:12 in the video, we see an example of perpendicular lines not remaining perpendicular after a transformation.

However, parallel lines do always stay in the same direction, as seen in the video. 

If you want to see a rigorous proof of this, click below.  
% will give a counterexample to show that area is not preserved.  Suppose you have a square with the bottom and left edges given by the vectors $e_1$ and $e_2$ respectively.  This square has area 1.  If we have a transformation that scales by a factor of 2 in both the $x$ and $y$ direction, then the new square has sides given by the vectors $2e_1$ and $2e_2$.  This image has area 4.  We can also see this from the video.\\



\begin{edXshowhide}{Parallel Lines Stay Parallel}

Note that we can treat each point, $(x,y)$, as a vector in standard position $\left[\begin{array}{c}  x\\ y   \end{array} \right]$.  Now, suppose $\ell_1$ and $\ell_2$ are two parallel lines.  If $\ell_1$ consists of points $p_1=\left[\begin{array}{c}  x\\ mx+b_1   \end{array} \right]$ then $\ell_2$ consists of points $p_2=\left[\begin{array}{c}  x\\ mx+b_2   \end{array} \right]$.  We will find the image of two points on each line and use these points to show the slope of the image lines, $T(\ell_i)$, are the same.   If $T$ is a linear transformation with standard matrix $A=\left[\begin{array}{cc} a & c\\ b& d   \end{array} \right]$, then $T(\ell_1)$ contains the points $T(\left[\begin{array}{c}  0\\ b_1   \end{array} \right])=\left[\begin{array}{cc} a & c\\ b& d   \end{array} \right]\left[\begin{array}{c}  0\\ b_1   \end{array} \right]=\left[\begin{array}{c}  cb_1\\ db_1   \end{array} \right]$, and $T(\left[\begin{array}{c}  1\\ m+b_1   \end{array} \right])=\left[\begin{array}{cc} a & c\\ b& d   \end{array} \right]\left[\begin{array}{c}  1\\ m+b_1   \end{array} \right]=\left[\begin{array}{c}  a+cm+cb_1\\ b+dm+db_1   \end{array} \right]$.  Thus, $T(\ell_1)$ has slope $\frac{b+dm+db_1-db_1}{a+cm+cb_1-cb_1}=\frac{b+dm}{a+cm}$.  Similarly $T(\ell_2)$ contains the points $\left[\begin{array}{c}  cb_2\\ db_2  \end{array} \right]$ and $\left[\begin{array}{c} a+cm+cb_2\\ b+dm+db_2  \end{array} \right]$, so $T(\ell_2)$ has slope $\frac{b+dm+db_1-db_1}{a+cm+cb_1-cb_1}=\frac{b+dm}{a+cm}$.  Therefore, we see that the image of the two lines have the same slope.

\end{edXshowhide}

}

\endedxproblem

\endedxvertical












