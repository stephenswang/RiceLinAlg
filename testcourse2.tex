\documentclass[12pt]{article}

\usepackage{edXpsl}	% edX
\usepackage{amsmath, amsthm, amsfonts, amssymb, color, mathrsfs, comment}
\usepackage{cancel}

%------------------------------------------
\parindent=0pt
\parskip=1ex

\begin{document}
%-----------------------------------------------%

\input{macros_edX.tex} %custom macros

\def\defaultproblemattributes{attempts="1" showanswer="attempted" rerandomize="per_student"}

% edXcourse: {course_number}{course display_name}[optional arguments like semester]
\begin{edXcourse}{Test}{Test LinAlg}[semester="2018_Summer" info_sidebar_name="Other Documents" start="2018-01-11T12:00" end="2019-12-18T18:00" course_image="rice-logo.jpg" display_coursenumber="TestLinAlg" course_organization="TestRice" graceperiod="1800 seconds" invitation_only="true" allow_anonymous="false" mobile_available="true"  org="Rice"]
 
% \begin{edXchapter}{Getting started}[url_name="cyca" start="2015-08-23T18:00"]
%  

% \def\edxbaseoutputname{edxtutorial}

% \input{overview/edxTutorial.tex}
 

% \endedxsequential
 
% \end{edXchapter} 






\begin{edXchapter}{Basis and Dimension}[url_name="block3" start="2018-01-11T16:00"]






\def\edxbaseoutputname{hw1}

\beginedxsequential{Homework 1}{due="2019-12-13T14:15" graded="true" format="Exercises"}






\beginedxvertical{Page One}

\beginedxtext{Preliminaries}


\endedxtext

\endedxvertical

\beginedxvertical{Introduction}


% \beginedxproblem{Find the Matrix}{\dpa1}

% Suppose $T: \R^3\rightarrow \R^2$ is given by 
% \[T\left( \left[ \begin{array}{c} a_1 \\ a_2\\ a_3 \end{array} \right] \right) =\left[ \begin{array}{c} 2a_1-a_3 \\ -a_1 + a_2 \end{array} \right] \] 
% for all vectors $

% What is $T(e_3)$?  

% \input{vectorentry.tex}

% \edXabox{type="custom" cfn="VectorEntry" expect="[[5],[-1],[3]]"}


% \edXsolution{ 

% }


% \endedxproblem

5\times 7matrix, columns span $\R^5$, then all solutions sets to $Ax= v$ are translates of 

\endedxvertical








% \input{comingsoon.tex}




\endedxsequential




\end{edXchapter}

\end{edXcourse}
\end{document}
