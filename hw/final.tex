

\beginedxvertical{Page One}


\beginedxtext{Row-Reduction Tools}

You may look back at the course material while you take this exam.  

You should feel free to use technology to do row reduction of matrices.  All other computations (including matrix multiplication and inversion) should be done by hand.  

Here are a couple of websites that will do row reduction for you:

\href{https://tinyurl.com/dmokjt}{Linear Algebra Toolkit}

\href{https://www.wolframalpha.com/}{Wolfram Alpha}



\endedxtext

\beginedxproblem{Allowable Resources}{\dpa2}

On this final exam, students are allowed to use technology to do which of the following operations?  Check all that apply.  

\edXabox{type="oldmultichoice" expect="Row reduce a matrix" options="Row reduce a matrix","Multiply matrices","Invert a matrix","Inner products"}

\edXsolution{ Read the text above!
}

\endedxproblem


\endedxvertical

\beginedxvertical{Final Page 1}




\beginedxproblem{Systems of Linear Equations}{\dpa1}

True or False: A system of 8 linear equations in 6 variables cannot have a unique solution.  

\edXabox{expect="False" options="True","False"}

True or False: A system of 6 linear equations in 8 variables cannot have a unique solution.  

\edXabox{expect="True" options="True","False"}


\edXsolution{ 

}

\endedxproblem





\beginedxproblem{Row-Reduced Matrix}{\dpa1}

Suppose $T: \R^5 \rightarrow \R^4$ is a linear transformation with standard matrix $A$.  I am not going to tell you what $A$ is,
but I will tell you that when you row-reduce $A$, the result is the matrix
\[R = 
\left[ \begin{array}{ccccc}    
1 & 2 & 0 & 0 & 3 \\
0 & 0 & 1 & 0 & -1 \\
0 & 0 & 0 & 1 & 0 \\
0 & 0 & 0 & 0 & 0
\end{array}
\right]
 \]

Answer the following questions about $T$ and/or $A$.  If there is not enough information to tell (e.g., because you only know $R$, not $A$),
select "Not enough information."

Is $A$ invertible?  
\edXabox{expect="No" options="Yes","No","Not enough information"}

Are the columns of $A$ linearly dependent?
\edXabox{expect="Yes" options="Yes","No","Not enough information"}

Is $T$ onto?
\edXabox{expect="No" options="Yes","No","Not enough information"}

Does the equation $Ax = \veco$ have a unique solution?
\edXabox{expect="No" options="Yes","No","Not enough information"}

Is the equation $Ax = \left[ \begin{array}{c} 3 \\ 1 \\ 0 \\ 1 \end{array} \right]$ consistent?
\edXabox{expect="Not enough information" options="Yes","No","Not enough information"}


\edXsolution{ 
  
}
 
\endedxproblem



\endedxvertical

\beginedxvertical{Final Page 2}

\beginedxproblem{Row-Reduced Matrix Continued}{\dpa5}

Take the same set-up as the previous problem.  Suppose $T: \R^5 \rightarrow \R^4$ is a linear transformation with standard matrix $A$. When you row-reduce $A$, the result is the matrix
\[R = 
\left[ \begin{array}{ccccc}    
1 & 2 & 0 & 0 & 3 \\
0 & 0 & 1 & 0 & -1 \\
0 & 0 & 0 & 1 & 0 \\
0 & 0 & 0 & 0 & 0
\end{array}
\right]
 \]

What is the rank of $T$?  

\edXabox{type="numerical" expect="3"} 

Find a basis for the kernel of $T$.  
Enter the list of vectors forming the basis below, separated by semicolons.  For instance, 
to enter the list $\left\{\left[\begin{array}{c} 1 \\ 0 \\ 1
\end{array} \right]; \left[\begin{array}{c} 1 \\ 2 \\ 3
\end{array} \right] \right\}$, type <1,0,1>;<1,2,3>.  

%requires new grading program


\input{solutionsetcheck5.tex}

\edXabox{type="custom" cfn="spangrader" expect="<-2,1,0,0,0>;<-3,0,1,0,1>"}

\edXsolution{ 
  
}
 
\endedxproblem



\beginedxproblem{Linear Transformations}{\dpa3}


Let $S: \R^3 \rightarrow \R^2$ be a linear transformation with 
\[
S(e_1) = \left[ \begin{array}{c} -1 \\ 4  \end{array} \right],
S(e_2) = \left[ \begin{array}{c} 2 \\ 1  \end{array} \right],
S(e_3) = \left[ \begin{array}{c} 3 \\ 3  \end{array} \right]. \]


Let $T: \R^2 \rightarrow \R^4$ be a linear transformation with 
$T(e_1) = \left[ \begin{array}{c} 3 \\ -1 \\ 0 \\ 1 \end{array} \right]$ 
and $T(e_2) = \left[ \begin{array}{c} 1 \\ 0 \\ 1 \\ -2 \end{array} \right].$

What is the standard matrix for $T\circ S$?  
 

\input{matrixentry.tex}


\edXabox{type="custom" cfn="MatrixEntry" expect="[[1,7,12],[1,-2,-3],[4,1,3],[-9,0,-3]]"}


\edXsolution{
}


\endedxproblem



\beginedxproblem{Invertible?}{\dpa1}

True or False: If $A$ is a $6\times 6$ matrix, and none of its columns are a scalar multiple of another,
then $A$ must be invertible.  

\edXabox{expect="False" options="True","False"}

True or False: If $A$ and $B$ are $6\times 6$ matrices, and $AB = I_6$, then $BA = I_6$.  

\edXabox{expect="True" options="True","False"}


\edXsolution{ 

}

\endedxproblem



\beginedxproblem{Invert a Matrix}{\dpa3}

Let $A$ be the matrix
\[ 
\left[ 
\begin{array}{ccc} 1& 1 &  2 \\ 2 & 1 & -3 \\ 1 & -1 & -11 
\end{array} \right] \]

Enter its inverse, if it has one.  If it does not have
an inverse, enter [[0]].  

\input{matrixentry3.tex}


\edXabox{type="custom" cfn="MatrixEntry" expect="[[14,-9,5],[-19,13,-7],[3,-2,1]]"}

\edXsolution{

We can check that $A$ row reduces to the identity.  When we row reduce the extended 
\[\left[ 
\begin{array}{ccc:ccc} 1& 1 &  2 & 1 & 0 & 0 \\ 2 & 1 & -3 & 0 & 1 & 0 \\ 1 & -1 & -11 & 0 & 0 & 1 
\end{array} \right],
\]
we obtain
\[\left[ 
\begin{array}{ccc:ccc} 1& 0 &  0 & 14 & -9 & 5 \\ 0 & 1 & 0 & -19 & 13 & -7 \\ 0 & 0 & 1 & 3 & -2 & 1 
\end{array} \right],
\]
so the inverse is
\[\left[ 
\begin{array}{ccc} 14 & -9 & 5 \\  -19 & 13 & -7 \\ 3 & -2 & 1 
\end{array} \right].
\]
}


\endedxproblem



\endedxvertical


\beginedxvertical{Final Page 3}

\beginedxproblem{Subspace?}{\dpa1}

True or False: If $A$ is a $10\times 12$ matrix and $v \in \R^{10}$, and the 
equation $Ax = v$ has infinitely many solutions, then the set of solutions to that equation forms
a subspace of $\R^{12}$.  

\edXabox{expect="False" options="True","False"}

\edXsolution{ 

}

\endedxproblem


\beginedxproblem{Coordinates}{\dpa3}


\begin{center}
\includesvg[450]{h5coords}
\end{center}

Let $\mathcal{B} = \{v_1; v_2\}$ be as pictured; this is a basis of $\R^2$.  

Given the vector $w$ in the image, what is $[w]_{\mathcal{B}}$?  

\input{vectorentry.tex}


\edXabox{type="custom" cfn="VectorEntry" expect="[[5],[3]]"}


\edXsolution{
We can write $w$ as a linear combination of the basis vectors in order:
$w = 5v_1 + 3v_2$.  Hence $[w]_{\mathcal{B}} =\left[\begin{array}{c}  5  \\ 3 \end{array} \right]$.  
}

\endedxproblem


\beginedxproblem{Span}{\dpa2}

Let $W$ be the span of the vectors 
\[
\left[ \begin{array}{c} 1 \\ 2 \\ 1 \\ 4 \\ 3\end{array} \right] ; 
\left[ \begin{array}{c} 2 \\ 3 \\ 3 \\ 1 \\ -1\end{array} \right] ; 
\left[ \begin{array}{c} 0 \\ -2 \\ 2 \\ -3 \\ 0 \end{array} \right] ;
\left[ \begin{array}{c} -1 \\ -7 \\ 4 \\ -6 \\ 4\end{array} \right] ;
\left[ \begin{array}{c} 3 \\ 4 \\ 5 \\ 9 \\ 9\end{array} \right]
 \]

What is  the dimension of $W$?

\edXabox{type="numerical" expect="3"} 

\edXsolution{
If $A$ is the matrix with these five vectors as columns, then the image of $A$ is the same as the span of the five vectors.
Hence the desired dimension is the rank of $A$.  When we row reduce $A$, we obtain 3 pivots, so the rank is 3.  
}

\endedxproblem



\beginedxproblem{Rank}{\dpa2}

Suppose that $T: \R^6 \rightarrow \R^7$ is a linear transformation. Suppose further that we can find two distinct vectors $v_1,v_2 \in \R^6$ such that $T(v_1)$ and $T(v_2)$ are both equal to the same non-zero vector $w \in R^7$.    

What is the strongest conclusion we can make about the rank of $T$?  


\edXinline{The rank of $T$ is at least}\edXabox{type="numerical" expect="1" inline="1"} 

\edXinline{and at most} \edXabox{type="numerical" expect="5" inline="1"}.  

\edXsolution{
Since $T(v_1) = T(v_2)$, we know that $T$ is not into.  Therefore the nullity of $T$ is at least 1.  Since the rank of $T$ plus its nullity must 
equal 6, the rank is at most 5.  We also know that $w$ is a non-zero vector in the image of $T$.  Hence the image of $T$ is a non-zero
subspace and has dimension at least 1.  Therefore the rank of $T$ is at least 1.  
}

\endedxproblem





\endedxvertical

\beginedxvertical{Final Page 4}



\beginedxproblem{Subspace Dimension}{\dpa2}

Let $V$ be a space with an inner product, and let $v\in V$ be a fixed non-zero vector.  Define $W$ to be
the set of all vectors $w\in V$ with the property that $\langle v,w\rangle = 0$.  

On Homework 4 you proved that $W$ is a subspace of $V$.  In this problem, we'll find its dimension.  

Suppose that $V$ has dimension $n$.  Define the function 
$T$ with domain $V$ by $T(w) = \langle v, w\rangle$.  $T$ is a linear transformation (you are encouraged to prove this on your own) with
what codomain?  

\edXabox{type="multichoice" expect="$\R$" options="$V$","$W$","$\R^n$","$\R$","Something else"}

What is the rank of $T$ (your answer can be in terms of $n$)?  

\edXabox{type="formula" expect="1" samples="n@1:5#5" feqin="1" tolerance=".01"}

Note that $W$ is exactly the kernel of $T$.  What is the dimension of $W$?    

\edXabox{type="formula" expect="n-1" samples="n@1:5#5" feqin="1" tolerance=".01"}

\edXsolution{ 

}
 
\endedxproblem




\beginedxproblem{Closest Vector}{\dpa3}

Let $W$ be the subspace of $\R^4$ spanned by the vectors 
$\left[\begin{array}{c} 
1 \\ 1 \\ -1 \\ -1 \end{array} \right]$ 
and 
$\left[\begin{array}{c} 
3 \\ 1 \\ 1 \\ 7 \end{array} \right].$

What is the closest element of $W$ to the vector 
$v =  \left[\begin{array}{c} 
3 \\ 4 \\ 3 \\ -8 \end{array} \right]?$


\input{vectorentry2.tex}


\edXabox{type="custom" cfn="VectorEntry" expect="[[1],[2],[-3],[-6]]"}

\edXsolution{ 

	
}


\endedxproblem



\beginedxproblem{Least Squared Quadratic}{\dpa3}


Consider all functions of the form $f(x) = ax^2 + c$, where $a$ and $c$ are real numbers.  
Find a function of that form which achieves
the least squared vertical distance to the data points $(-1, 1), (0,1), (1, 2),$ and $(2, -1)$. 

Remember to type * for multiplication.

\edXabox{type="formula" expect="-x^2/18+4/3" samples="x@1:5#5" feqin="1" tolerance=".01"}



\edXsolution{ 

}

\endedxproblem


\endedxvertical







