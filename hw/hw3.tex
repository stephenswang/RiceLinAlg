

\beginedxvertical{Page One}


\beginedxtext{Computational Tools}

For this homework, you should feel free to use technology to do row reduction, multiplication, and inversion of matrices.  All other
computations should still be done by hand, for now.  

Here are a couple of websites that will do some of this for you:

\href{https://tinyurl.com/3xr6ed}{Linear Algebra Toolkit}

\href{https://www.wolframalpha.com/}{Wolfram Alpha}



\endedxtext


\endedxvertical

\beginedxvertical{Homework Page 1}


\beginedxproblem{Five-Element Subspace?}{\dpa1}

True or false: There is no subspace of $\R^5$ that contains exactly five elements.  

\edXabox{expect="True" options="True","False"}

\edXsolution{ If there is even a single non-zero vector in a subspace, then every scalar multiple of that vector must also be in the
subspace.  That is an infinite  number of vectors.  
}
 
\endedxproblem

\beginedxproblem{Line as Subspace?}{\dpa1}

True or false: Any line in $\R^3$ is a subspace of $\R^3$.  

\edXabox{expect="False" options="True","False"}

\edXsolution{ If the line does not contain the zero vector, it is not a subspace.
}
 
\endedxproblem


\beginedxproblem{Subspace of Matrices?}{\dpa1}

Let $M_{2\times 2}$ denote the vector space  of $2\times 2$ real matrices, and let $W$ be the 
set of symmetric $2\times 2$ real matrices.  (Recall that a matrix is symmetric if it is its own
transpose.)


True or false: $W$ is a subspace of $M_{2\times 2}$.  Prove your answer, or come up with a counterexample showing that it is not.  

\edXabox{expect="I have proven it" options="I have proven it","I found a counterexample"}

\edXsolution{ Let's check the three criteria for being a subspace.  

The zero matrix is symmetric. 

If $A$ and $B$ are both symmetric, then $(A+B)^t = A^t + B^t = A + B,$ so $A+B$ is symmetric.  Hence
$W$ is closed under addition.  

If $A$ is symmetric and $c$ is a scalar, then $(cA)^t = cA^t = cA,$ so $cA$ is symmetric.  Hence
$W$ is closed under scalar multiplication.  

Therefore $W$ is a subspace of $M_{2\times 2}.$  

}
 
\endedxproblem




\endedxvertical

\beginedxvertical{Homework Page 2}


\beginedxproblem{Basis}{\dpa1}

Is the following list a basis of $\R^3$?  
\[ \left\{  \left[ \begin{array}{c} 1 \\ 3 \\ -3 \end{array} \right] ; 
\left[ \begin{array}{c} 2 \\ 4 \\ -2 \end{array} \right] ;
\left[ \begin{array}{c} 5 \\ 3 \\ 9 \end{array} \right]  \right\} \]



\edXabox{expect="No" options="Yes","No"}

\edXsolution{ 
When we row-reduce the $3\times 3$ matrix with these three vectors as its columns, we do not get the identity matrix.  
Therefore this list is not a basis of $\R^3$. 
}

\endedxproblem

\beginedxproblem{Quick Picks}{\dpa2}

Without doing any row-reduction, you should be able to pick out five of the 
following lists of vectors that are {\keyb{\bf{not}}} a basis  of $\R^4$.  Which five?  

\begin{itemize}

\item 
List A:
$\left\{ \left[ \begin{array}{c} 1 \\ 3 \\ 1 \\4 \end{array} \right]; 
\left[ \begin{array}{c} 1 \\ 1 \\ 0 \\ 1 \end{array} \right]; 
\left[ \begin{array}{c} 2 \\ -1 \\ 3 \\ 1 \end{array} \right] \right\}
$


\item
List B:
$\left\{\left[ \begin{array}{c} 1 \\ 2 \\ 3 \\ 4 \end{array} \right] ; 
\left[ \begin{array}{c} 3 \\ 1 \\ 3 \\ 1\end{array} \right] ; 
\left[ \begin{array}{c} 0 \\ 0 \\ 0 \\ 0\end{array} \right] ; 
\left[ \begin{array}{c} 2 \\ 5 \\ 7  \\ 10 \end{array} \right] \right\} $



\item 
List C:
$\left\{
\left[ \begin{array}{c} 1 \\ 2 \\ 3 \\ 4  \end{array} \right] ;
\left[ \begin{array}{c} 1 \\ 2 \\ 3 \\ 5 \end{array} \right] ;
\left[ \begin{array}{c} 1 \\ 3 \\ 4 \\ 5  \end{array} \right] ; 
\left[ \begin{array}{c} 2 \\ 3 \\ 4 \\ 5 \end{array} \right] \right\} $

\item 
List D:
$\left\{\left[ \begin{array}{c} 1 \\ 2 \\ 1 \\ 4\end{array} \right] ; 
\left[ \begin{array}{c} 2 \\ 13 \\ 3 \\ 7\end{array} \right] ; 
\left[ \begin{array}{c} 0 \\ 1 \\ 5 \\ -3\end{array} \right] ;
\left[ \begin{array}{c} 15 \\ -2 \\ 6 \\ 2\end{array} \right] ;
\left[ \begin{array}{c} 5 \\ 3 \\ 8 \\ 11\end{array} \right] \right\} $

\item
List E:
$\left\{\left[ \begin{array}{c} 0 \\ 1 \\ 1 \\ 1 \end{array} \right] ; 
\left[ \begin{array}{c} 2 \\ 0 \\ 1 \\ 1 \end{array} \right] ;
\left[ \begin{array}{c} 0 \\ 3 \\ 0 \\ 1 \end{array} \right] ;
\left[ \begin{array}{c} 0 \\ 0 \\ 4 \\ 0 \end{array} \right] \right\}
$

\item 
List F:
$\left\{\left[ \begin{array}{c} 1 \\ 1 \\ 1 \\ 1  \end{array} \right] ;
\left[ \begin{array}{c} 5 \\ 2 \\ -8 \\ 4  \end{array} \right] ;
\left[ \begin{array}{c} -2 \\ -2 \\ -2 \\ -2 \end{array} \right] ; 
\left[ \begin{array}{c} 1 \\ 3 \\ 1 \\ 5 \end{array} \right] \right\} $




\item
List G:
$\left\{\left[ \begin{array}{c} 1 \\ 0 \\ 1 \\ 1 \end{array} \right] ; 
\left[ \begin{array}{c} 3 \\ 0 \\ 1 \\ 0 \end{array} \right] ;
\left[ \begin{array}{c} 1 \\ 0 \\ 2 \\ 4 \end{array} \right] ;
\left[ \begin{array}{c} 2 \\ 0 \\ -3 \\ 5 \end{array} \right] \right\}
$

\end{itemize}




\edXabox{type="oldmultichoice" expect="A","B","D","F","G" options="A","B","C","D","E","F","G"}

\edXsolution{ 
Any basis of $\R^4$ must have exactly 4 vectors.  That rules out A and D as bases.  

A list containing the zero vector cannot be linearly independent.  This rules out B. 
Similarly, list F has one vector which is a scalar multiple of another, so it too is linearly
dependent.  

Finally, list G does not span $\R^4$, as you cannot get a vector with a non-zero second
coordinate as a linear combination.  So G is not a basis of $\R^4$ either.   
}

\endedxproblem



\endedxvertical

\beginedxvertical{Homework Page 3}


\beginedxproblem{Polynomial Subspace}{\dpa2}

Let $W$ be the set of real polynomials of degree at most 6 which have no terms of odd degree (that is, polynomials 
such as $3t^6 + 2t^2 -1$ that do not include a $t^5, t^3,$ or $t^1$ term).  This is a subspace of $\mathbb{P_6}$.  
What is the dimension of $W$?   

\edXabox{type="numerical" expect="4"} 

\edXsolution{
$W$ is all polynomials of the form $a_0 + a_2t^2 + a_4t^4 + a_6t^6$; i.e., the span of $\{1; t^2; t^4; t^6\}$.  This
is a linearly independent list which spans $W$, so it is a basis of $W$.  The dimension of $W$ is thus the number of
elements of that list. 
}

\endedxproblem

\beginedxproblem{Subspace}{\dpa2}

Suppose $W$ is a subspace of $\R^5$, and suppose that $e_1, e_3 \in W$, but $e_2, e_4, e_5 \notin W$.  

What is the strongest statement we can make about the dimension of $W$?

\edXinline{The dimension of $W$ is at least}\edXabox{type="numerical" expect="2" inline="1"} 

\edXinline{and at most} \edXabox{type="numerical" expect="4" inline="1"}.  


\edXsolution{
$\{e_1; e_3\}$ is a  linearly independent set inside $W$, so the dimension of $W$ is at least 2.  However, it can be bigger than that; for instance, if $W$ is the span of $\{e_1; e_3; e_2-e_4; e_4-e_5\}$ it would have dimension 4 (one can check that those four vectors are linearly independent) and satisfy the requirements.   

$W$ cannot have dimension 5, however, since the only subspace of $\R^5$ that has dimension
5 is $\R^5$ itself.  Thus the maximum is 4.  
}

\endedxproblem


\beginedxproblem{Kernel}{\dpa2}

Suppose that $T: \R^9 \rightarrow \R^4$ is a linear transformation which is not onto.  

What is the strongest conclusion we can make about the kernel of $T$?  

\edXinline{The dimension of $\mathrm{Ker}(T)$ is at least}\edXabox{type="numerical" expect="6" inline="1"} 

\edXinline{and at most} \edXabox{type="numerical" expect="9" inline="1"}.  

\edXsolution{
Since $T$ is not onto, the rank of $T$ cannot be 4.  Thus the rank is at most 3.  Since the rank plus the nullity must equal 9, this means
the nullity is at least 6.  The kernel is a subspace of $\R^9$, so its dimension is at most 9.  
}

\endedxproblem


\beginedxproblem{Rank}{\dpa2}

Suppose that $T: \R^5 \rightarrow \R^7$ is a linear transformation. Suppose further that we can find two vectors $v,w \in \R^5$, neither 
of which is a scalar multiple of the other, such that $T(v) = T(w) = \veco$.  

What is the strongest conclusion we can make about the rank of $T$?  


\edXinline{The rank of $T$ is at least}\edXabox{type="numerical" expect="0" inline="1"} 

\edXinline{and at most} \edXabox{type="numerical" expect="3" inline="1"}.  

\edXsolution{
We have $v$ and $w$ as linearly independent vectors in $\mathbb{Ker}(T).$  Therefore the nullity of $T$ is at least 2.  Since the nullity plus
rank must equal 5, that means the rank is at most 3.  
}

\endedxproblem






\beginedxproblem{Subspace Basis 1}{\dpa1}

True or False: If $V$ is finite-dimensional, and $W$ is a subspace of $V$, then every basis of $V$ must
contain a basis of $W$.  

\edXabox{expect="False" options="True","False"}

\edXsolution{
This is not true.  For instance, the standard basis of $\R^2$ does not contain a basis of the subspace
spanned by $\left[ \begin{array}{c} 1 \\ 1 \end{array} \right]$.  
}

\endedxproblem

\beginedxproblem{Subspace Basis 2}{\dpa1}

True or False: If $V$ is finite-dimensional, and $W$ is a subspace of $V$, then every basis of $W$ is contained in some basis of $V$.  

\edXabox{expect="True" options="True","False"}

\edXsolution{
True.  Any basis of $W$ is linearly independent, and we showed that any linearly independent set in 
a finite-dimensional space $V$ can 
be extended to basis of $V$.  
}

\endedxproblem



\endedxvertical

\beginedxvertical{Homework Page 4}


\beginedxproblem{Kernel Basis}{\dpa5}

Suppose $T: \R^4 \rightarrow \R^4$ is a linear transformation with standard matrix
\[A = 
\left[ \begin{array}{cccc}    
3 & 1 & 1 & -2  \\
-1 & -2 & 3 & -1  \\
1 & -3 & 7 & -4  \\
-3 & 4 & -11 & 7 
\end{array}
\right]
 \]



Find a basis for the kernel of $T$.  
Enter the list of vectors forming the basis below, separated by semicolons.  For instance, 
to enter the list $\left\{\left[\begin{array}{c} 1 \\ 0 \\ 1
\end{array} \right]; \left[\begin{array}{c} 1 \\ 2 \\ 3
\end{array} \right] \right\}$, type <1,0,1>;<1,2,3>.  

%requires new grading program



\begin{edXscript}

def VectorEntry(ans):
    import numpy as np
    import ast
    test={'ok':False}
    try:
        ans=ans.split(";")
        hold=[]
        for a in ans:
            a=a.split(",")
            sub=[]
            for i in range(len(a)):
                if i==0:
                    sub.append(float(a[i][1:]))
                elif i==len(a)-1:
                    sub.append(float(a[i][:-1]))
                else:
                    sub.append(float(a[i]))
            assert len(sub)==4
            hold.append(np.array(sub))
        if len(hold)==2:
            test['ok']=True
        else:
            test['msg']='You gave the wrong number of vectors'        
    except AssertionError:
        test['msg']='One of your vectors is not a 4-tuple'
    except:
        test['msg']='Wrong input format'
    return test  
def span2(a,b):
    import numpy as np
    import ast
    non=np.nonzero(a)
    first=non[0][0]
    den=a[first]
    num=b[first]
    quot=num//den
    if np.array_equal(np.multiply(a,quot),b):
        return(-1)
    else:
        return(1)
        
def spangrader(expect,ans):
    import numpy as np
    import ast
    mata=np.array([[3,1,1,-2],[-1,-2,3,-1],[1,-3,7,-4],[-3,4,-11,7]])
    res={'ok':False}
    if VectorEntry(ans)['ok']!=True:
        return(VectorEntry(ans))
    else:
        ans=ans.split(";")
        hold=[]
        for a in ans:
            a=a.split(",")
            sub=[]
            for i in range(len(a)):
                if i==0:
                    sub.append(float(a[i][1:]))
                elif i==len(a)-1:
                    sub.append(float(a[i][:-1]))
                else:
                    sub.append(float(a[i]))
            hold.append(np.array(sub))
        check_len=True
        check_zero=True
        check_lin=False
        veco=np.array([0,0,0,0])
        new=[]
        for item in hold:
            mul=np.dot(mata,item)
            if np.array_equal(np.dot(mata,item),np.array([0,0,0,0])) and not np.array_equal(item,veco):
                new.append(item)
            elif not np.array_equal(np.dot(mata,item),np.array([0,0,0,0])):
                check_zero=False
                res['msg']='At least one of the vectors is not in the kernel'
        if (len(new)==1 or len(new)==0) and check_zero==True:
            res['msg']='The set of vectors given does not span the kernel'
        else:
            for i in range(len(new)-1):
                for j in range(i+1,len(new)):
                    if span2(new[i],new[j])==1:
                        check_lin=True
                        break
            if check_lin==True:
                res['ok']=True
            elif check_zero==True and len(new)>=2:
                res['msg']='You may have linearly dependent vectors'
        return res

\end{edXscript}


\edXabox{type="custom" cfn="spangrader" expect="<1,-1,0,1>;<-1,2,1,0>"}



\edXsolution{ 
 We row reduce $A$ and obtain
\[\left[ \begin{array}{cccc}    
1 & 0 & 1 & -1  \\
0 & 1 & -2 & 1  \\
0 & 0 & 0 & 0  \\
0 & 0 & 0 & 0  
\end{array}
\right] 
\]

Therefore, solutions to $Ax = \veco$ must be of the form 
\[
\left[ \begin{array}{c} x_1 \\ x_2 \\ x_3 \\ x_4
\end{array} \right] 
= 
\left[ \begin{array}{c} -x_3 + x_4 \\ 2x_3 - x_4 \\ x_3 \\ x_4
\end{array} \right] 
= x_3 \left[ \begin{array}{c} -1 \\ 2 \\ 1 \\ 0
\end{array} \right] + x_4 \left[ \begin{array}{c} 1 \\ -1 \\ 0 \\ 1
\end{array} \right],\]
where $x_3$ and $x_4$ are free variables.  Hence the list
\[ \left\{ \left[ \begin{array}{c} -1 \\ 2 \\ 1 \\ 0
\end{array} \right]; \left[ \begin{array}{c} 1 \\ -1 \\ 0 \\ 1
\end{array} \right] \right\} \]
forms a basis of the kernel of $T$.  
}
 
\endedxproblem


\beginedxproblem{Image Basis}{\dpa5}

With $T$ as in the previous problem, 
find a basis for the image of $T$.  
Enter the list of vectors forming the basis below, separated by semicolons.  For instance, 
to enter the list $\left\{\left[\begin{array}{c} 1 \\ 0 \\ 1
\end{array} \right]; \left[\begin{array}{c} 1 \\ 2 \\ 3
\end{array} \right] \right\}$, type <1,0,1>;<1,2,3>.  

%requires new grading program



\begin{edXscript}

def VectorEntry(ans):
    import numpy as np
    import ast
    test={'ok':False}
    try:
        ans=ans.split(";")
        hold=[]
        for a in ans:
            a=a.split(",")
            sub=[]
            for i in range(len(a)):
                if i==0:
                    sub.append(float(a[i][1:]))
                elif i==len(a)-1:
                    sub.append(float(a[i][:-1]))
                else:
                    sub.append(float(a[i]))
            assert len(sub)==4
            hold.append(np.array(sub))
        if len(hold)==2:
            test['ok']=True
        else:
            test['msg']='You gave the wrong number of vectors'        
    except AssertionError:
        test['msg']='One of your vectors is not a 4-tuple'
    except:
        test['msg']='Wrong input format'
    return test  
def span2(a,b):
    import numpy as np
    import ast
    non=np.nonzero(a)
    first=non[0][0]
    den=a[first]
    num=b[first]
    quot=num//den
    if np.array_equal(np.multiply(a,quot),b):
        return(-1)
    else:
        return(1)
        
def spangrader(expect,ans):
    import numpy as np
    import ast
    mata=np.array([[1,1,1,1],[1,2,-1,0]])
    res={'ok':False}
    if VectorEntry(ans)['ok']!=True:
        return(VectorEntry(ans))
    else:
        ans=ans.split(";")
        hold=[]
        for a in ans:
            a=a.split(",")
            sub=[]
            for i in range(len(a)):
                if i==0:
                    sub.append(float(a[i][1:]))
                elif i==len(a)-1:
                    sub.append(float(a[i][:-1]))
                else:
                    sub.append(float(a[i]))
            hold.append(np.array(sub))
        check_len=True
        check_zero=True
        check_lin=False
        veco=np.array([0,0,0,0])
        new=[]
        for item in hold:
            mul=np.dot(mata,item)
            if np.array_equal(np.dot(mata,item),np.array([0,0])) and not np.array_equal(item,veco):
                new.append(item)
            elif not np.array_equal(np.dot(mata,item),np.array([0,0])):
                check_zero=False
                res['msg']='At least one of the vectors is not in the image'
        if (len(new)==1 or len(new)==0) and check_zero==True:
            res['msg']='The set of vectors given does not span the image'
        else:
            for i in range(len(new)-1):
                for j in range(i+1,len(new)):
                    if span2(new[i],new[j])==1:
                        check_lin=True
                        break
            if check_lin==True:
                res['ok']=True
            elif check_zero==True and len(new)>=2:
                res['msg']='You may have linearly dependent vectors'
        return res

\end{edXscript}


\edXabox{type="custom" cfn="spangrader" expect="<3,-1,1,-3>;<1,-2,-3,4>"}



\edXsolution{ 
We again row reduce $A$ as in the previous solution, obtaining 
 \[\left[ \begin{array}{cccc}    
1 & 0 & 1 & -1  \\
0 & 1 & -2 & 1  \\
0 & 0 & 0 & 0  \\
0 & 0 & 0 & 0  
\end{array}
\right].
\]
With pivots in the first two columns, we know that the first two columns of $A$ will form a basis of 
the image of $T$.  
}
 
\endedxproblem

\endedxvertical



\beginedxvertical{Discussion}


\beginedxtext{Discussion}

Feel free to discuss the homework questions in the forum.  Before creating a new post, please first check to see if there is already a question
or discussion addressing your topic.  

Also, please do not give away the answers!  

\endedxtext

\edXdiscussion{Discuss Homework 3}{discussion_category="Homework" discussion_topic="HW3" discussion_id="HW3"}

\endedxvertical





