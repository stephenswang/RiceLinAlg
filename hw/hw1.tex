

\beginedxvertical{Page One}

\beginedxtext{Row-Reduction Tools}

For this homework, you should feel free to use technology to do row reduction of matrices.  All other
computations should still be done by hand, for now.  

Here are a couple of websites that will do row reduction for you:

\href{https://tinyurl.com/dmokjt}{Linear Algebra Toolkit}

\href{https://www.wolframalpha.com/}{Wolfram Alpha}



\endedxtext


\endedxvertical

\beginedxvertical{Homework Page 1}



\beginedxproblem{Parabola}{\dpa3}

Is there a quadratic equation of the form
$y = a + bx + cx^2 $ satisfied by the four points 
$(-2,11), (0,4), (1,2),$ and $(2,1)$?


Enter the correct formula in the form a + b*x + c*x^2  (either exactly or to two decimal places).  Remember to use * for multiplication.  If there is no such equation, enter the number 0.  

\edXinline{$y=$ }
\edXabox{type="formula" expect="4-2.5*x+0.5*x^2" samples="x@1:5#5" feqin="1" inline="1" tolerance=".01"}

\edXsolution{
For the parabola to pass through these points, we would need the following equations to be satisfied:
\[
\begin{array}{rcccccc}
11  & = & a & - & 2b & + & 4c \\
4  & = & a & + & 0b & + & 0c \\
2  & = & a & + & 1b & + & 1c \\
1  & = & a & + & 2b & + & 4c 
\end{array}
\]

This is a system of four linear equations in three variables $a,b,c,$ represented by the augmented matrix
\[ \left[ \begin{array}{ccc:c}
1&-2&4&11 \\
1&0&0&4\\
1&1&1&2\\
1&2&4&1
\end{array} \right].\]
When we row-reduce this matrix, we obtain
\[ \left[ \begin{array}{ccc:c}
1&0&0&4 \\
0&1&0&-2.5\\
0&0&1&0.5\\
0&0&0&0
\end{array} \right].\]
This tells us that when $a=4, b=-2.5, c=0.5,$ the equations are all satisfied.  Therefore
$y=4 -2.5x + 0.5x^2$ passes through all four points.  

}

\endedxproblem


\beginedxproblem{Parabola 2}{\dpa3}

Is there a quadratic equation of the form
$y = a + bx + cx^2 $ satisfied by the four points 
$(-2,11), (0,4), (1,2),$ and $(3,0)$?


Enter the correct formula in the form $a + b*x + c*x^2$, either exactly or to two decimal places.  (Remember to use * for multiplication.)  If there is no such equation, enter the number 0.  

\edXinline{$y=$ }
\edXabox{type="formula" expect="0" samples="x@1:5#5" feqin="1" inline="1" tolerance=".01"}

\edXsolution{


For the parabola to pass through these points, we would need the following equations to be satisfied:
\[
\begin{array}{rcccccc}
11  & = & a & - & 2b & + & 4c \\
4  & = & a & + & 0b & + & 0c \\
2  & = & a & + & 1b & + & 1c \\
0  & = & a & + & 3b & + & 9c 
\end{array}
\]

This is a system of four linear equations in three variables $a,b,c,$ represented by the augmented matrix
\[ \left[ \begin{array}{ccc:c}
1&-2&4&11 \\
1&0&0&4\\
1&1&1&2\\
1&3&9&0
\end{array} \right].\]
When we row-reduce this matrix, we obtain
\[ \left[ \begin{array}{ccc:c}
1&0&0&0 \\
0&1&0&0\\
0&0&1&0\\
0&0&0&1
\end{array} \right].\]
This is an inconsistent system, so there is no quadratic passing through all four points.
}

\endedxproblem

\beginedxproblem{Plate Temperature}{\dpa3}

An elliptical metal plate is shown below.  We can measure the temperature at six points along its edge.  

\begin{center}
\includesvg[450]{h1plate}
\end{center}

We wish to determine the temperature at some internal points, labelled A, B, C, D.  Assuming that the temperature at each
of the points is the average of the temperatures at its four neighboring points (as given by the
dotted lines), set up a system of linear equations for the temperatures, and then solve it.  

\edXinline{$A=$ }
\edXabox{type="numerical" expect="8" feqin="1" inline="1" tolerance=".01"}

\edXinline{$B=$ }
\edXabox{type="numerical" expect="12" feqin="1" inline="1" tolerance=".01"}

\edXinline{$C=$ }
\edXabox{type="numerical" expect="9" feqin="1" inline="1" tolerance=".01"}

\edXinline{$D=$ }
\edXabox{type="numerical" expect="10" feqin="1" inline="1" tolerance=".01"}

\edXsolution{
The condition that the four points' temperatures be equal to the average of nearby temperatures are:
\[
\begin{array}{rcl}
A & = & \frac{1}{4}(B + C + 3 + 8) \\
B & = & \frac{1}{4}(A + C + D + 21) \\
C & = & \frac{1}{4}(A + B + D + 6) \\
D & = & \frac{1}{4}(B + C + 12 + 7) 
\end{array}
\]

We can manipulate this to get the equations
\[
\begin{array}{ccccccccc}
4A & - & B & -&  C & & & = & 11\\
-A  & + & 4B & - & C & - & D & = & 21 \\
-A  & - & B & + & 4C & - & D & = & 6 \\
  & - & B & - & C & + & 4D & = & 7  
\end{array}
\]
which translates to the augmented matrix 
\[
\left[
\begin{array}{cccc:c}
4 & -1 & -1 & 0 &  11\\
-1 & 4 & -1 & -1 & 21 \\
-1 &-1 &  4 & -1  & 6 \\
 0 & -1 & -1 & 4 & 7  
\end{array}\right].
\]
This row reduces to 
\[
\left[\begin{array}{cccc:c}
1 & 0 & 0 & 0 &  8\\
0& 1 & 0 & 0 & 12 \\
0 &0 &  1 & 0  & 9 \\
 0 & 0 & 0 & 1 & 10  
\end{array}\right].
\]

So $(A,B,C,D) = (8,12,9,10)$.  
}

\endedxproblem


% \beginedxproblem{Find the Matrix}{\dpa1}

% Suppose $T: \R^3\rightarrow \R^2$ is given by 
% \[T\left( \left[ \begin{array}{c} a_1 \\ a_2\\ a_3 \end{array} \right] \right) =\left[ \begin{array}{c} 2a_1-a_3 \\ -a_1 + a_2 \end{array} \right] \] 
% for all vectors $

% What is $T(e_3)$?  

% \input{vectorentry.tex}

% \edXabox{type="custom" cfn="VectorEntry" expect="[[5],[-1],[3]]"}


% \edXsolution{ 

% }


% \endedxproblem

% K5\times 7matrix, columns span $\R^5$, then all solutions sets to $Ax= v$ are translates of 



\endedxvertical

\beginedxvertical{Homework Page 2}


\beginedxproblem{Traffic}{\dpa5}

A set of one-way roads and intersections is shown below.  The rate of traffic (measured in cars per hour) is shown for some roads.  
Assume that at each of the six intersections, 
the flow of traffic into the intersection equals the flow of traffic
out of the intersection.  This gives a set of linear equations in the seven unknown traffic rates.  


\begin{center}
\includesvg[450]{h1traffic}
\end{center}


Solve the system in full generality.  Assuming that all rates are non-negative, what is the minimum possible value of $x_6$?  

\edXabox{type="numerical" expect="60" feqin="1" tolerance=".01"}

\edXsolution{
The condition that flow in equals flow out at each intersection gives us six equations:
\[
\begin{array}{ccccccc}
& & x_1 & = & x_2 & + & 100 \\
x_2 &+& 50 & = & x_3 &+& x_7 \\
&& x_3 & = & x_4 &+& 130 \\
x_4 &+& 150 & = & x_5 & & \\
x_5 &+& x_7 & = & x_6 &+& 90 \\
x_6 &+& 120 & = & x_1 &&  \\
\end{array}
\]
This can be rearranged into a system of linear equations with the following augmented matrix:
\[
\left[
\begin{array}{ccccccc:c}
1 & -1 & 0 & 0 & 0 & 0 & 0 & 100 \\
0 & 1 & -1 & 0 & 0 & 0 & -1 & -50 \\
0 & 0 & 1 & -1 & 0 & 0 & 0 & 130 \\
0 & 0 & 0 & 1 & -1 & 0 & 0 & -150 \\
0 & 0 & 0 & 0 & 1 & -1 & 1 & 90 \\
-1 & 0 & 0 & 0 & 0 & 1 & 0 & -120 
\end{array}
\right],
\]
which row-reduces to 
\[
\left[
\begin{array}{ccccccc:c}
1 & 0 & 0 & 0 & 0 & -1 & 0 & 120 \\
0 & 1 & 0 & 0 & 0 & -1 & 0 & 20 \\
0 & 0 & 1 & 0 & 0 & -1 & 1 & 70 \\
0 & 0 & 0 & 1 & 0 & -1 & 1 & -60 \\
0 & 0 & 0 & 0 & 1 & -1 & 1 & 90 \\
0 & 0 & 0 & 0 & 0 & 0 & 0 & 0 
\end{array}
\right].
\]
We see that $x_6$ and $x_7$ are free variables.  However, the fourth line tells that 
$x_4 - x_6 + x_7 = -60$, or $x_6 = x_4 + x_7 + 60$.  If all variables are non-negative, then
$x_6 \ge 60$.  By setting $x_6 = 60$ and $x_7= 0$, all other variables will be non-negative, 
so 60 is indeed possible, and is therefore the answer.  
}
\endedxproblem



\beginedxproblem{Matrix-Vector Multiplication}{\dpa3}



Let  $u,v,w$ be three vectors in $\R^4$ with the property that $4u - 3v + 2w = \veco$.  
Let $A$ be the $4\times 2$ matrix whose columns are $u$ and $v$ (in that order).  

Find a solution to the equation $Ax = w$.  

\input{vectorentry.tex}


\edXabox{type="custom" cfn="VectorEntry" expect="[[-2],[1.5]]"}

\edXsolution{ 
We can solve the given equation for $w$: 
\[ w = -2u + 1.5v.\]  
This is a linear combination of the columns of $A$; it is $-2$ times the first column of $A$ plus
$1.5$ times the second column of $A$.  By definition, this is the same as 
the matrix-vector product $A\left[ \begin{array}{c} -2 \\ 1.5 \end{array}\right].$  

Therefore $x = \left[ \begin{array}{c} -2 \\ 1.5 \end{array}\right]$ is a solution to $Ax = w.$  
}


\endedxproblem






\beginedxproblem{Find a Solution Set}{\dpa7}


Let  \[A = \left[ \begin{array}{cccc} 1 & -2 & 0 & 3\\ 
1 & -2 & 2 & -1 \\ 2 & -4 & 1 & 4 \end{array} \right].\]  Find a list
of vectors whose span is the set of solutions to $Ax = \veco$.  

Enter the list of vectors below, separated by semicolons.  For instance, 
to enter the list $\left\{\left[\begin{array}{c} 1 \\ 0 \\ 1
\end{array} \right]; \left[\begin{array}{c} 1 \\ 2 \\ 3
\end{array} \right] \right\}$, type <1,0,1>;<1,2,3>.  

%requires new grading program



\begin{edXscript}

def VectorEntry(ans):
    import numpy as np
    import ast
    test={'ok':False}
    try:
        ans=ans.split(";")
        hold=[]
        for a in ans:
            a=a.split(",")
            sub=[]
            for i in range(len(a)):
                if i==0:
                    sub.append(float(a[i][1:]))
                elif i==len(a)-1:
                    sub.append(float(a[i][:-1]))
                else:
                    sub.append(float(a[i]))
            assert len(sub)==4
            hold.append(np.array(sub))
        test['ok']=True
    except AssertionError:
        test['msg']='One of your vectors is not a 4-tuple'
    except:
        test['msg']='Wrong input format'
    return test  
def span2(a,b):
    import numpy as np
    import ast
    non=np.nonzero(a)
    first=non[0][0]
    den=a[first]
    num=b[first]
    quot=num//den
    if np.array_equal(np.multiply(a,quot),b):
        return(-1)
    else:
        return(1)
        
def spangrader(expect,ans):
    import numpy as np
    import ast
    mata=np.array([[1,-2,0,3],[1,-2,2,-1],[2,-4,1,4]])
    res={'ok':False}
    if VectorEntry(ans)['ok']!=True:
        return(VectorEntry(ans))
    else:
        ans=ans.split(";")
        hold=[]
        for a in ans:
            a=a.split(",")
            sub=[]
            for i in range(len(a)):
                if i==0:
                    sub.append(float(a[i][1:]))
                elif i==len(a)-1:
                    sub.append(float(a[i][:-1]))
                else:
                    sub.append(float(a[i]))
            hold.append(np.array(sub))
        check_len=True
        check_zero=True
        check_lin=False
        veco=np.array([0,0,0,0])
        new=[]
        for item in hold:
            mul=np.dot(mata,item)
            if np.array_equal(np.dot(mata,item),np.array([0,0,0])) and not np.array_equal(item,veco):
                new.append(item)
            elif not np.array_equal(np.dot(mata,item),np.array([0,0,0])):
                check_zero=False
                res['msg']='At least one of the vectors is not in the solution set'
        if len(new)==1 or len(new)==0 and check_zero==True:
            res['msg']='The set of vectors given does not span the solution set'
        else:
            for i in range(len(new)-1):
                for j in range(i+1,len(new)):
                    if span2(new[i],new[j])==1:
                        check_lin=True
                        break
            if check_lin==True:
                res['ok']=True
            elif check_zero==True and len(new)>=2:
                res['msg']='You may be missing a vector'
        return res

\end{edXscript}


\edXabox{type="custom" cfn="spangrader" expect="<2,1,0,0>;<-3,0,2,1>"}


\edXsolution{ 
We row reduce the matrix $A$ to obtain
\[
\left[ \begin{array}{cccc} 1 & -2 & 0 & 3 \\  0 & 0 & 1 & -2 \\ 0 & 0 & 0 & 0 \end{array} \right] \] 

We have 2 free variables, $x_2$ and $x_4$. For the non-free variables, we get
$x_1 = 2x_2 - 3x_4$ and $x_3 = 2x_4$.  

Hence the general solution is \[
\left[ \begin{array}{c} x_1 \\ x_2 \\ x_3 \\ x_4 \end{array}\right] = 
\left[ \begin{array}{c} 2x_2 - 3x_4 \\ x_2 \\ 2x_4 \\ x_4 \end{array}\right] = 
x_2 \left[ \begin{array}{c} 2 \\ 1 \\ 0 \\ 0 \end{array}\right] + 
x_4 \left[ \begin{array}{c} -3 \\ 0 \\ 2 \\ 1 \end{array}\right],
\]
for any $x_2,x_4 \in \R$.  Hence the set of solutions is the span of  
$\left\{ \left[ \begin{array}{c} 2 \\ 1 \\ 0 \\ 0 \end{array} \right]; 
\left[ \begin{array}{c} -3 \\ 0 \\ 2 \\ 1 \end{array} \right]\right\}.$

(Note that this is not the only list that spans the solution set, but it is the one that is simplest to find.)

}


\endedxproblem


\endedxvertical

\beginedxvertical{Homework Page 3}

\beginedxproblem{Lists of Vectors}{\dpa1}

For each of the following lists of vectors in $\R^3$, say whether the list spans $\R^3$, is linearly independent, both, or neither. 



\[ \left\{ 
\left[ \begin{array}{c} 1 \\ 2 \\ 3\end{array} \right] ; 
\left[ \begin{array}{c} 5 \\ 2 \\ 0\end{array} \right] \right\} \]

\edXabox{type="multichoice" expect="The list is linearly independent but does not span $\R^3$" options="The list spans $\R^3$ but is not linearly independent","The list is linearly independent but does not span $\R^3$","The list spans $\R^3$ and is linearly independent","The list neither spans $\R^3$ nor is linearly independent"}

\[ \left\{ 
\left[ \begin{array}{c} 1 \\ 1 \\ 0\end{array} \right] ; 
\left[ \begin{array}{c} 1 \\ 2 \\ 1\end{array} \right] ;
\left[ \begin{array}{c} 0 \\ 1 \\ 3\end{array} \right] 
\right\} \]

\edXabox{type="multichoice" expect="The list spans $\R^3$ and is linearly independent" options="The list spans $\R^3$ but is not linearly independent","The list is linearly independent but does not span $\R^3$","The list spans $\R^3$ and is linearly independent","The list neither spans $\R^3$ nor is linearly independent"}

\[ \left\{ 
\left[ \begin{array}{c} 1 \\ 0 \\ 2\end{array} \right] ; 
\left[ \begin{array}{c} 0 \\ 2 \\ -1\end{array} \right] ;
\left[ \begin{array}{c} 2 \\ -2 \\ 5\end{array} \right] ;
\left[ \begin{array}{c} -1 \\ 6 \\ -5\end{array} \right] 
\right\} \]

\edXabox{type="multichoice" expect="The list neither spans $\R^3$ nor is linearly independent" options="The list spans $\R^3$ but is not linearly independent","The list is linearly independent but does not span $\R^3$","The list spans $\R^3$ and is linearly independent","The list neither spans $\R^3$ nor is linearly independent"}


\edXsolution{ 
In the first example, we see two vectors.  Two vectors are not enough to span $\R^3$.  A pair of vectors
is linearly independent if and only if neither is a scalar multiple of the other -- so we can see by 
inspection that this list is linearly independent.  


For the other lists, we need to do more calculation.  We can write the vectors as the columns of a matrix, and row-reduce.  If there is a
pivot in every row, we know that the vectors will span $\R^3$, and if there is a pivot in every column,
the vectors will be linearly independent.  We see both when we apply this to the second list, so 
that list is both linearly independent and spans $\R^3$.  

For the third list, after row-reduction we obtain a $3\times 4$ matrix that only has two pivots, so
this list does not span $\R^3,$ nor is it linearly independent.  (In fact, a list of four vectors in
$\R^3$ can never be linearly independent!)
}

\endedxproblem


\beginedxproblem{Linear Transformation}{\dpa4}



\begin{center}
\includesvg[450]{h1lintrans}
\end{center}

The vectors $u,v,w \in \R^2$ are given in the picture above.  
Recall that $\mathbb{P}$ is the space of all polynomials.  
Suppose that $T: \R^2 \rightarrow \mathbb{P}$ is a linear transformation, and that
$T(u) = t^2 - 1$ and $T(v) = -2t + 3$.  What is $T(w)$?  


\edXabox{type="formula" expect="-t^2 -4*t + 7" samples="t@1:5#10" tolerance=".01" feqin="1" size="50"}


\edXsolution{Note from the picture that $w = -u + 2v.$  Hence \[
\begin{array}{rcl}
T(w) &  =  & T(-u+2v) \\
& = & T(-u) + T(2v) \\
& = & -T(u) + 2T(v) \\
& = & -t^2 -4t + 7. 
\end{array}
\]

}

\endedxproblem

\beginedxproblem{Equal Matrix-Vector Products}{\dpa1}

True or False: If $A$ is a $6\times 4$ matrix whose columns are linearly independent, and 
$Au = Av$ for vectors $u,v \in \R^4$, then $u=v$.  


\edXabox{expect="True" options="True","False"}


\edXsolution{ 
Let $Au = Av = w$.  Since the columns of $A$ are linearly independent, the equation $Ax = w$ can have at most 
one solution.  Since $u$ and $v$ are both solutions to this equation, they must be equal.  
}

\endedxproblem


\endedxvertical

\beginedxvertical{Homework Page 4}




\beginedxproblem{Standard Matrix}{\dpa3}

For all $v\in \R^2$, let $T(v)$ be the result when $v$ is reflected across the diagonal line $y=-x$ and
then tripled in length.    
$T$ is a linear transformation from $\R^2$ to $\R^2$.  What is its standard matrix?
 
\input{matrixentry.tex}

\edXabox{type="custom" cfn="MatrixEntry" expect="[[0,-3],[-3,0]]"}


\edXsolution{ We can see that the reflection of $e_1$ is $-e_2$,  thus $T(e_1) = -3e_2$.  Therefore the first column of the standard matrix is $-3e_2$.  
Similarly, the reflection of $e_2$ is $-e_1$, so we have $T(e_2) = -3e_1$, and the second column of the standard matrix is $-3e_1$.  
Therefore the standard matrix is 
$\left[ 
\begin{array}{cc} 0& -3 \\ -3 & 0 \end{array} \right].$
}


\endedxproblem


\beginedxproblem{Into and Onto}{\dpa1}

Suppose the linear transformation $T: \R^5\rightarrow \R^3$ has a standard matrix $A$ which row reduces
to have 3 pivots.  What can we conclude about $T$?  


\edXabox{type="multichoice" expect="$T$ is onto but not into" options="$T$ is onto but not into","$T$ is into but not onto","$T$ is both into and onto","$T$ is neither into nor onto"}


Suppose the linear transformation $T: \R^5\rightarrow \R^3$ has a standard matrix $A$ which row reduces
to have 2 pivots.  What can we conclude about $T$?  


\edXabox{type="multichoice" expect="$T$ is neither into nor onto" options="$T$ is onto but not into","$T$ is into but not onto","$T$ is both into and onto","$T$ is neither into nor onto"}


Suppose the linear transformation $T: \R^3\rightarrow \R^5$ has a standard matrix $A$ which row reduces
to have 3 pivots.  What can we conclude about $T$?  


\edXabox{type="multichoice" expect="$T$ is into but not onto" options="$T$ is onto but not into","$T$ is into but not onto","$T$ is both into and onto","$T$ is neither into nor onto"}


\edXsolution{ 
We know that $T$ is into if and only if its standard matrix row-reduces to have a pivot in every column,
and $T$ is onto if and only if its standard matrix row-reduces to have a pivot in every row.  

In the first example, we have a $3\times 5$ matrix with a pivot in every row but not every column, 
so $T$ is onto but not into.  (In fact, a linear transformation from $\R^5$ to $\R^3$ can never be into.)

In the second example, we have a $3\times 5$ matrix with only two pivots, therefore it does not have one in every row nor in every column.  
Thus $T$ is neither into nor onto. 

Finally, we have a $5\times 3$ matrix with a pivot in every column, but not every row.  Thus 
$T$ is into but not onto.  (In fact, a linear transformation from $\R^3$ to $\R^5$ can never be onto.)
}

\endedxproblem


\endedxvertical



\beginedxvertical{Discussion}


\beginedxtext{Discussion}

Feel free to discuss the homework questions in the forum.  Before creating a new post, please first check to see if there is already a question
or discussion addressing your topic.  

Also, please do not give away the answers!  

\endedxtext

\edXdiscussion{Discuss Homework 1}{discussion_category="Homework" discussion_topic="HW1" discussion_id="HW1"}

\endedxvertical

