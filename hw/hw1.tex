

\beginedxvertical{Page One}

\beginedxtext{Row-Reduction Tools}

For this homework, you should feel free to use technology to do row reduction of matrices.  All other
computations should still be done by hand, for now.  

Here are a couple of websites that will do row reduction for you:

\href{https://tinyurl.com/3xr6ed}{Linear Algebra Toolkit}

\href{https://www.wolframalpha.com/}{Wolfram Alpha}


\endedxtext


\endedxvertical

\beginedxvertical{Homework Page 1}



\beginedxproblem{Parabola}{\dpa3}

Is there a quadratic equation of the form
$y = a + bx + cx^2 $ satisfied by the four points 
$(-2,11), (0,4), (1,2),$ and $(2,1)$?


Enter the correct formula in the form a + b*x + c*x^2  (either exactly or to two decimal places).  Remember to use * for multiplication.  If there is no such equation, enter the number 0.  

\edXinline{$y=$ }
\edXabox{type="formula" expect="4-2.5*x+0.5*x^2" samples="x@1:5#5" feqin="1" inline="1" tolerance=".01"}

\edXsolution{
}

\endedxproblem


\beginedxproblem{Parabola 2}{\dpa3}

Is there a quadratic equation of the form
$y = a + bx + cx^2 $ satisfied by the four points 
$(-2,11), (0,4), (1,2),$ and $(3,0)$?


Enter the correct formula in the form $a + b*x + c*x^2$, either exactly or to two decimal places.  (Remember to use * for multiplication.)  If there is no such equation, enter the number 0.  

\edXinline{$y=$ }
\edXabox{type="formula" expect="0" samples="x@1:5#5" feqin="1" inline="1" tolerance=".01"}

\edXsolution{
}

\endedxproblem

\beginedxproblem{Plate Temperature}{\dpa3}

An elliptical metal plate is shown below.  We can measure the temperature at six points along its edge.  

\begin{center}
\includesvg[450]{h1plate}
\end{center}

We wish to determine the temperature at some internal points, labelled A, B, C, D.  Assuming that the temperature at each
of the points is the average of the temperatures at its four neighboring points (as given by the
dotted lines), set up a system of linear equations for the temperatures, and then solve it.  

\edXinline{$A=$ }
\edXabox{type="numerical" expect="8" feqin="1" inline="1" tolerance=".01"}

\edXinline{$B=$ }
\edXabox{type="numerical" expect="12" feqin="1" inline="1" tolerance=".01"}

\edXinline{$C=$ }
\edXabox{type="numerical" expect="9" feqin="1" inline="1" tolerance=".01"}

\edXinline{$D=$ }
\edXabox{type="numerical" expect="10" feqin="1" inline="1" tolerance=".01"}

\edXsolution{
}

\endedxproblem


% \beginedxproblem{Find the Matrix}{\dpa1}

% Suppose $T: \R^3\rightarrow \R^2$ is given by 
% \[T\left( \left[ \begin{array}{c} a_1 \\ a_2\\ a_3 \end{array} \right] \right) =\left[ \begin{array}{c} 2a_1-a_3 \\ -a_1 + a_2 \end{array} \right] \] 
% for all vectors $

% What is $T(e_3)$?  

% \input{vectorentry.tex}

% \edXabox{type="custom" cfn="VectorEntry" expect="[[5],[-1],[3]]"}


% \edXsolution{ 

% }


% \endedxproblem

% K5\times 7matrix, columns span $\R^5$, then all solutions sets to $Ax= v$ are translates of 



\endedxvertical

\beginedxvertical{Homework Page 2}


\beginedxproblem{Traffic}{\dpa5}

A set of roads and intersections is shown below.  The rate of cars per hour is shown for some roads.  
Assume that at each of the six intersections, 
the flow of traffic into the intersection equals the flow of traffic
out of the intersection.  This gives a set of linear equations in the seven unknown traffic rates.  


\begin{center}
\includesvg[450]{h1traffic}
\end{center}


Solve the system in full generality.  Assuming that all rates are non-negative, what is the minimum possible value of $x_7$?  

\edXabox{type="numerical" expect="60" feqin="1" tolerance=".01"}

\edXsolution{
}

\endedxproblem


\endedxvertical

\beginedxvertical{Homework Page 3}


\beginedxproblem{Cube to Identity}{\dpa7}

Find a $2\times 2$ matrix $A$ which is not the identity matrix, but which has the property
that $A^3 = I_2$.  Enter your answer to within two decimal places.  

(Hint: Don't try to solve for the entries of $A$ with algebra.  Think about what you might want $A$ to do geometrically first.)


\begin{edXscript}
def cubing(expect,ans):
    import ast
    import numpy as np
    res={'ok':False}
    if MatrixEntry(expect,ans)['ok']!=True:
        res['msg']=(MatrixEntry(expect,ans))['msg']
    else:
        hold=np.array(eval(ans))
        mat=hold
        I=np.identity(2)
        mat_c=np.linalg.matrix_power(mat,3)
        diff1=np.subtract(mat_c,I)
    
        adiff1=np.absolute(diff1)
        check1=(adiff1>0.02).any()
        diff2=np.subtract(mat,I)
        adiff2=np.absolute(diff2)
        check2=(adiff2>0.01).any()
    

        if check2==False:
            res['msg']='Answer too close to the identity'
            
        elif check1==True:
            res['msg']='Answer does not cube to the identity'
            
        else:
            res['ok']=True
    return res
  
def MatrixEntry(expect, ans):
    import ast
    import numpy as np
    ret= {'ok':False}
    
    
    try:
        list_ans = ast.literal_eval(ans)
        list_expect = ast.literal_eval(expect)
        matrix_ans = np.matrix(list_ans)
        matrix_expect = np.matrix(list_expect)
        if matrix_ans.shape != matrix_expect.shape:
            ret['msg'] = 'Wrong shape of matrix'
        
        else:
            ret['ok'] = True
    except SyntaxError:
        ret['msg'] = 'Wrong input format'
    return ret
\end{edXscript}


\edXabox{type="custom" cfn="cubing" expect="[[-0.5,-0.866],[0.866,-0.5]]"}

\edXsolution{

}


\endedxproblem







\endedxvertical






