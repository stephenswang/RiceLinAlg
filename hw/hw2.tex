

\beginedxvertical{Page One}


\beginedxtext{Row-Reduction Tools}

For this homework, you should feel free to use technology to do row reduction of matrices.  All other
computations should still be done by hand, for now.  

Here are a couple of websites that will do row reduction for you:

\href{https://tinyurl.com/dmokjt}{Linear Algebra Toolkit}

\href{https://www.wolframalpha.com/}{Wolfram Alpha}




\endedxtext


\endedxvertical

\beginedxvertical{Homework Page 1}


\beginedxproblem{Find a Matrix}{\dpa3}

Suppose that $T: \R^2 \rightarrow \R^2$ has standard matrix 
$\left[ 
\begin{array}{cc} 1& -3 \\ 2 & 1 \end{array} \right],$
and $S: \R^3 \rightarrow \R^2$ has standard matrix
$\left[ 
\begin{array}{ccc} 0& 2 &  -1 \\ 1 & 0 & 2 \end{array} \right].$

What is the standard matrix for $T\circ S$?  

 
\input{matrixentry.tex}

\edXabox{type="custom" cfn="MatrixEntry" expect="[[-3,2,-7],[1,4,0]]"}


\edXsolution{
The standard matrix for $T\circ S$ is the standard matrix for $T$ times the standard matrix for $S$.  
Doing this multiplication results in $\left[ 
\begin{array}{ccc} -3& 2 &  -7 \\ 1 & 4 & 0 \end{array} \right].$
}


\endedxproblem

\beginedxproblem{Products}{\dpa2}

Suppose $A = \left[ 
\begin{array}{ccc} 0& 2 &  -1 \\ 1 & 0 & 2 \end{array} \right]$ and $B = \left[ 
\begin{array}{cccc} 1&  1& 2 &  -1 \\ 1 & 4 & 3 & 2 \end{array} \right].$
Click all of the following expressions which are defined.

\edXabox{type="oldmultichoice" expect="$B^tA$","$(BB^tAA^t)^2$" options="$AB$","$(BA)^t$","$B^tA$","$BA^t$","$B^tA^t$","$(BB^tAA^t)^3$"}

\edXsolution{
The only things we need to pay attention to are the sizes of $A$ and $B$; we see that $A$ is $2\times 3$ and
$B$  is $2\times 4$.  Hence $AB$ is not defined, nor is $BA$, and hence not $(BA)^t$.  $A^t$ is $3\times 2$
and $B^t$ is $4\times 2$, so $B^tA$ is defined but $BA^t$ is not, nor is $B^tA^t$.  

Finally, $BB^t$ is $2\times 2$, as is $AA^t$.  Therefore $BB^tAA^t$ is also $2\times 2$ and can be cubed.  
}

\endedxproblem


\beginedxproblem{Cube to Identity}{\dpa7}

Find a $2\times 2$ matrix $A$ which is not the identity matrix, but which has the property
that $A^3 = I_2$.  Enter your answer to within two decimal places.  

(Hint: Don't try to solve for the entries of $A$ with algebra.  Think about what you might want $A$ to do geometrically first.)


\begin{edXscript}
def cubing(expect,ans):
    import ast
    import numpy as np
    res={'ok':False}
    if MatrixEntry(expect,ans)['ok']!=True:
        res['msg']=(MatrixEntry(expect,ans))['msg']
    else:
        hold=np.array(eval(ans))
        mat=hold
        I=np.identity(2)
        mat_c=np.linalg.matrix_power(mat,3)
        diff1=np.subtract(mat_c,I)
    
        adiff1=np.absolute(diff1)
        check1=(adiff1>0.02).any()
        diff2=np.subtract(mat,I)
        adiff2=np.absolute(diff2)
        check2=(adiff2>0.01).any()
    

        if check2==False:
            res['msg']='Answer too close to the identity'
            
        elif check1==True:
            res['msg']='Answer does not cube to the identity'
            
        else:
            res['ok']=True
    return res
  
def MatrixEntry(expect, ans):
    import ast
    import numpy as np
    ret= {'ok':False}
    
    
    try:
        list_ans = ast.literal_eval(ans)
        list_expect = ast.literal_eval(expect)
        matrix_ans = np.matrix(list_ans)
        matrix_expect = np.matrix(list_expect)
        if matrix_ans.shape != matrix_expect.shape:
            ret['msg'] = 'Wrong shape of matrix'
        
        else:
            ret['ok'] = True
    except SyntaxError:
        ret['msg'] = 'Wrong input format'
    return ret
\end{edXscript}


\edXabox{type="custom" cfn="cubing" expect="[[-0.5,-0.866],[0.866,-0.5]]"}

\edXsolution{
Any such matrix $A$ would be the standard matrix for a linear transformation $T: \R^2
\rightarrow \R^2$ that has the property that $T \circ T \circ T$ is the identity map.  One
example of such a linear transformation would be a rotation by $120^{\circ}$.  The standard
matrix for that rotation is $\left[ \begin{array} {cc} -\frac{1}{2} & -\frac{\sqrt{3}}{2} \\
\frac{\sqrt{3}}{2} & -\frac{1}{2} \end{array} \right].$  
}


\endedxproblem


\endedxvertical

\beginedxvertical{Homework Page 2}


\beginedxproblem{Invert a Matrix}{\dpa3}

For each of the following matrices, enter its inverse, if it has one.  If it does not have
an inverse, enter [[0]].  

\input{matrixentry3.tex}

$\left[ 
\begin{array}{ccc} 1& 2 &  -1 \\ 1 & 0 & 2 \end{array} \right]$

\edXabox{type="custom" cfn="MatrixEntry" expect="[[0]]"}


$\left[ 
\begin{array}{ccc} 1& 3 &  -1 \\ 2 & 2 & 1 \\ 0 & 4 & -3 \end{array} \right]$

\edXabox{type="custom" cfn="MatrixEntry" expect="[[0]]"}


$\left[ 
\begin{array}{ccc} 1& 1 &  0 \\ 1 & 2 & -3 \\ 1 & 1 & 1 
\end{array} \right]$

\edXabox{type="custom" cfn="MatrixEntry" expect="[[5,-1,-3],[-4,1,3],[-1,0,1]]"}

\edXsolution{
The first matrix is not square, so it has no inverse.  

The second matrix is square.  However, when we row reduce it, we do not get the identity matrix.  Therefore it too
is non-invertible.

The last matrix does row reduce to the identity.  When we row reduce the extended 
\[\left[ 
\begin{array}{ccc:ccc} 1& 1 &  0 & 1 & 0 & 0 \\ 1 & 2 & -3 & 0 & 1 & 0 \\ 1 & 1 & 1 & 0 & 0 & 1 
\end{array} \right],
\]
we obtain
\[\left[ 
\begin{array}{ccc:ccc} 1& 0 &  0 & 5 & -1 & -3 \\ 0 & 1 & 0 & -4 & 1 & 3 \\ 0 & 0 & 1 & -1 & 0 & 1 
\end{array} \right],
\]
so the inverse is
\[\left[ 
\begin{array}{ccc} 5 & -1 & -3 \\  -4 & 1 & 3 \\ -1 & 0 & 1 
\end{array} \right].
\]
}


\endedxproblem


\beginedxproblem{Solve}{\dpa4}

Suppose that $T:\R^4\rightarrow \R^4$ is a linear transformation with standard matrix $A$.
Suppose further that we know that $A$ is invertible, and that
that \[
A\inv = \left[ 
\begin{array}{cccc} 1& 1 &  0 & a_1 \\ 1 & 2 & -3 & a_2 \\ 1 & 1 & 1 & a_3 \\ 7 & 0 & 1 & a_4 
\end{array} \right],
\]
for some scalars $a_1, a_2, a_3, a_4$.  

Solve the equation $T(v) = \left[ \begin{array}{c} 3 \\ 2 \\ 1 \\ 0 \end{array} \right]$.

\input{vectorentry.tex}


\edXabox{type="custom" cfn="VectorEntry" expect="[[5],[4],[6],[22]]"}


\edXsolution{
Solving $T(v) = w$ is equivalent to solving $Av = w$; since $A$ is invertible we know the solution must be
$v = A\inv w$.  Thus, we multiply $A\inv$ by $\left[ \begin{array}{c} 3 \\ 2 \\ 1 \\ 0 \end{array} \right],$
and obtain $\left[ \begin{array}{c} 5 \\ 4 \\ 6 \\ 22 \end{array} \right].$  (It does not matter that we don't
know the $a_i$ since they get multiplied by zero.)  
}

\endedxproblem

\endedxvertical

\beginedxvertical{Homework Page 3}

\beginedxproblem{Implies Invertible? 1}{\dpa1}


True or False: If a matrix $A$ has linearly independent columns, it must be invertible.  


\edXabox{expect="False" options="True","False"}


\edXsolution{ 
Not necessarily!  If $A$ is known to be square, then having linearly independent columns would be enough to imply
invertibility.  But we don't know that.  
}

\endedxproblem

\beginedxproblem{Implies Invertible? 2}{\dpa1}

True or False: If square matrices $A$ and $B$ have the property that $AB = I$, then $BA = I$.  

\edXabox{expect="True" options="True","False"}


\edXsolution{ 
If two {\keyb{square}} matrices multiply to be the identity, then they are necessarily 
inverses of one another.  This was proven as part of the Invertible Matrix Theorem. 
}

\endedxproblem

\beginedxproblem{Implies Invertible? 3}{\dpa1}

True or False: If the columns of an $n\times n$ matrix $A$ span $\R^n$, then $A^tA$ must be invertible.  

\edXabox{expect="True" options="True","False"}


\edXsolution{ 
Since $A$ is square, if its columns span $\R^n$, it must be invertible by the Invertible Matrix Theorem.
Therefore $A^t$ is invertible.  The product of two invertible matrices is invertible, so $A^tA$ is invertible.  
}

\endedxproblem

\beginedxproblem{Solutions?}{\dpa1}

True or False: If $A$ is a $4\times 4$ matrix, and the equation 
$Ax = e_1$ has exactly one solution, then 
$Ax = e_4$ must have at least one solution.  


\edXabox{expect="True" options="True","False"}


\edXsolution{ 
If $Ax = e_1$ has exactly one solution, then $A$ must be invertible by the Invertible Matrix Theorem.  Therefore
$Ax = v$ has a (unique) solution for every $v \in \R^4$.  
}

\endedxproblem



\endedxvertical

\beginedxvertical{Homework Page 4}

\beginedxproblem{Linear Transformation}{\dpa4}

Suppose that $T:\R^3\rightarrow \R^3$ is a linear transformation with the property
that \[
\begin{array}{rcl}
T\left(\left[ \begin{array}{c} 1 \\ 7 \\ 0\end{array} \right]\right) &=& e_1\\ 
T\left(\left[ \begin{array}{c} 3 \\ 2 \\ 1\end{array} \right]\right) &=& e_2\\ 
T\left(\left[ \begin{array}{c} 0 \\ 5 \\ 3\end{array} \right]\right) &=& e_3. 
\end{array}
\]

Find a vector $w$ such that $T(w) = \left[ \begin{array}{c} 1 \\ 3 \\ -2\end{array} \right]$.

\input{vectorentry2.tex}


\edXabox{type="custom" cfn="VectorEntry" expect="[[10],[3],[-3]]"}

Find a vector $w$ such that $T(w) = \left[ \begin{array}{c} 1000 \\ 100 \\ 10 \end{array} \right]$.



\edXabox{type="custom" cfn="VectorEntry" expect="[[1300],[7250],[130]]"}



\edXsolution{
Call the three given vectors $v_1, v_2, v_3,$ such that $T(v_i) = e_i$.  
Note that \[
\left[ \begin{array}{c} 1 \\ 3 \\ -2\end{array} \right] = 1e_1 + 3e_2 - 2e_3 = T(v_1 + 3v_2 - 2v_3).\]
Therefore 
\[w = v_1 + 3v_2 - 2v_3 = \left[ \begin{array}{c} 10 \\ 3 \\ -3\end{array} \right]  \]
is a solution to the first equation.  

Similarly, 
\[w = 1000v_1 + 100v_2 + 10v_3 = \left[ \begin{array}{c} 1300 \\ 7250 \\ 130\end{array} \right]  \]
is a solution to the second equation.  
}

\endedxproblem

\beginedxproblem{Invertible?}{\dpa1}

Use the same $T$ as in the previous problem. 

Will $T$ necessarily be invertible, will it necessarily be non-invertible, or is there not enough information?
Think about how you would prove your answer, or find a counterexample.

\edXabox{type="multichoice" expect="$T$ must be invertible" options="$T$ must be invertible","$T$ must be non-invertible","There is not enough information to determine"}

What is the significance of the $3\times 3$ matrix whose columns (in order) are $
\left[ \begin{array}{c} 1 \\ 7 \\ 0\end{array} \right], 
\left[ \begin{array}{c} 3 \\ 2 \\ 1\end{array} \right],$ and
$\left[ \begin{array}{c} 0 \\ 5 \\ 3\end{array} \right]$?

\edXabox{type="multichoice" expect="It is the standard matrix for $T\inv$" options="It is the standard matrix for $T$","It is the standard matrix for $T\inv$","Neither of the above"}


\edXsolution{
The argument in the previous problem will work no matter what vector is on the right side of the equation; $T(w) = u$ always 
has a solution, for every $u$.  In other words, $T$ is onto.  

By the Invertible Matrix Theorem, $T$ is therefore invertible.  

Note that 
\[
\begin{array}{rcl}  T\inv(e_1) &=& \left[ \begin{array}{c} 1 \\ 7 \\ 0\end{array} \right]\\
T\inv(e_2) &=& \left[ \begin{array}{c} 3 \\ 2 \\ 1\end{array} \right]\\
T\inv(e_3) &=& \left[ \begin{array}{c} 0 \\ 5 \\ 3\end{array} \right]
\end{array} 
\]
so these three vectors form the columns of the standard matrix for $T\inv$.  
}


\endedxproblem



\endedxvertical



\beginedxvertical{Discussion}


\beginedxtext{Discussion}

Feel free to discuss the homework questions in the forum.  Before creating a new post, please first check to see if there is already a question
or discussion addressing your topic.  

Also, please do not give away the answers!  

\endedxtext

\edXdiscussion{Discuss Homework 2}{discussion_category="Homework" discussion_topic="HW2" discussion_id="HW2"}

\endedxvertical





