

\beginedxvertical{Page One}

\beginedxtext{Preliminaries}





At the end of this sequence, and after some practice, you should be able to:

\begin{itemize}
\item Recognize the defining properties of an inner product.  
\item Be able to compute inner products.  
\item Be able to compute norms of vectors and distances between vectors.  
\item Determine when vectors are orthogonal.  
\end{itemize}


For time budgeting purposes, this sequence has 3 videos totaling 18 minutes, 
plus some questions.  




\endedxtext

\endedxvertical



\beginedxvertical{Dot Products}


\beginedxtext{The Dot Product}

So far in this course, we have only worked with two vector operations: vector addition and scalar 
multiplication.  In this chapter, we will introduce inner products.  The easiest example of an inner
product is the dot product in $\R^n$.  

{\keya{\bf{Definition.}}} If $v = \left[ \begin{array}{c} a_1 \\ a_2 \\ \vdots \\ a_n \end{array} \right]$
and $w = \left[ \begin{array}{c} b_1 \\ b_2 \\ \vdots \\ b_n \end{array} \right]$ are two vectors in
$\R^n$, we define the {\keyb{\bf{dot product}}} of $v$ and $w$ to be 
\[ v\dotprod w = a_1b_1 + a_2b_2 + \ldots + a_nb_n. \]

Often we will write the dot product as $\langle v, w\rangle$ instead of $v\dotprod w$.  

\endedxtext



% \beginedxproblem{What is it?}{\dpa1}

% If $v,w \in \R^5$, then $\langle v, w\rangle$ is an element of what set?

% \edXabox{type="multichoice" expect="$\R$" options="$\R$","$\R^5$","Neither of these"}

% \edXsolution{
% The dot product of two vectors in $\R^n$ is a scalar, in $\R$.  
% }

% \endedxproblem


\beginedxproblem{Calculate it}{\dpa2}

Let $v = \left[ \begin{array}{c} 2 \\ 1 \\ 0 \\ -2 \end{array} \right],$
and $w = \left[ \begin{array}{c} 4 \\ -7 \\ -3 \\ 6 \end{array} \right].$

What is $\langle v, w \rangle$?  

\edXabox{type="numerical" expect="-11" feqin="1"  tolerance=".01"}


\edXsolution{
We calculate $(2)(4) + (1)(-7) + (0)(-3) + (-2)(6) = -11.$  
}

\endedxproblem



\endedxvertical



\beginedxvertical{Dot Product Properties}


\beginedxproblem{Property 1}{\dpa2}
Suppose $v,w \in \R^n$, and $\langle v, w \rangle  = 5.$
What is $\langle w, v \rangle$?  

\edXabox{type="numerical" expect="5" feqin="1"  tolerance=".01"}


\edXsolution{
If $v = \left[ \begin{array}{c} a_1 \\ a_2 \\ \vdots \\ a_n \end{array} \right]$
and $w = \left[ \begin{array}{c} b_1 \\ b_2 \\ \vdots \\ b_n \end{array} \right]$, then 
\[ v\dotprod w = a_1b_1 + a_2b_2 + \ldots + a_nb_n = b_1a_1 + b_2a_2 + \ldots + b_na_n = 
w \dotprod v.  \]  
In other words, the dot product is symmetric, and is a commutative operation.  
}

\endedxproblem

\beginedxproblem{Property 2}{\dpa2}
Suppose $v,w \in \R^n$, and $\langle v, w \rangle  = 5.$
What is $\langle 3v, w \rangle$?  

\edXabox{type="numerical" expect="15" feqin="1"  tolerance=".01"}


\edXsolution{
If $v = \left[ \begin{array}{c} a_1 \\ a_2 \\ \vdots \\ a_n \end{array} \right]$
and $w = \left[ \begin{array}{c} b_1 \\ b_2 \\ \vdots \\ b_n \end{array} \right]$, then 
\[ \langle 3v, w \rangle = (3a_1)b_1 + (3a_2)b_2 + \ldots + (3a_n)b_n =
3\langle v, w \rangle.  \]  

More generally, for any scalar $c$, $\langle cv, w\rangle = c\langle v, w \rangle = \langle v, cw\rangle.$  
}

\endedxproblem


\beginedxproblem{Property 3}{\dpa2}
Let's consider the dot product of a vector with itself.  
If $v \in \R^3$, what is the smallest possible value of $\langle v, v\rangle$?  

\edXabox{type="numerical" expect="0" feqin="1"}

For what vector $v$ is that value of $\langle v, v\rangle$ attained?  

\input{vectorentry.tex}


\edXabox{type="custom" cfn="VectorEntry" expect="[[0],[0],[0]]"}

\edXsolution{
If $v = \left[ \begin{array}{c} a_1 \\ a_2 \\ a_3 \end{array} \right]$, then 
\[ v\dotprod v = a_1^2 + a_2^2 + a_3^2. \]  
This is always non-negative, and the only way it can equal zero is if all $a_i$ are zero, i.e., if $v= \veco$. 
}

\endedxproblem



\beginedxtext{Property 4}

One last property of the dot product is that it distributes over addition.  In other words, 
\[ \langle u+v, w \rangle = \langle u, w\rangle + \langle v, w \rangle \]
for all vectors $u,v,w \in \R^n$.


\endedxtext


\endedxvertical



\beginedxvertical{Inner Products}


\doedxvideo{Inner Products}{XSSNLiMSa0w}




\beginedxproblem{What is it?}{\dpa1}

If we have an inner product on $V$, and vectors $v,w \in V$, then $\langle v, w\rangle$ is...

\edXabox{expect="a scalar" options="a scalar","a vector in V","a matrix","none of these"}

\edXsolution{
The inner product of two vectors must be a scalar.  
}

\endedxproblem

\beginedxproblem{Calculate it 2}{\dpa3}

Let $f(x) = 1$ and $g(x) = x^3$.  
Using the inner product on the vector space $C[0,2]$ as described in the video, 
calculate $\langle f, g \rangle$.  

\edXabox{type="numerical" expect="4" feqin="1"  tolerance=".01"}


\edXsolution{
\[ \langle f, g \rangle = \int_0^2 f(x) g(x) \ dx = \int_0^2 x^3 \ dx = \left. \frac{x^4}{4} \right|_0^2 = 4.\]
}

\endedxproblem


\endedxvertical



\beginedxvertical{Inner Products Defined}

\beginedxtext{Inner Product Definition}

{\keya{\bf{Definition.}}} Given a vector space $V$ over $\R$, an {\keyb{\bf{inner product}}} on $V$ is any
rule that assigns a scalar $\langle v,w\rangle \in \R$ to any pair $v,w\in V$ which satisfies the following
properties:

\begin{itemize}
\item
$\langle v,w \rangle = \langle w,v\rangle$ for all $v,w \in V$.
\item
$\langle u+v,w \rangle = \langle u,w\rangle + \langle v,w\rangle $ for all $u,v,w \in V$.  
\item
$\langle av, w\rangle = a\langle v,w \rangle$ for all $v,w\in V$ and $a\in \R$.  
\item
$\langle v,v\rangle > 0$ for all non-zero $v \in V$.  

\end{itemize}

\endedxtext


\endedxvertical



\beginedxvertical{Magnitude}

\doedxvideo{Magnitude}{rNLDG-XjIGA}

\beginedxtext{Magnitude Definition}

{\keya{\bf{Definition.}}} Given a vector space $V$ with an inner product, the {\keyb{\bf{norm}}}
or {\keyb{\bf{magnitude}}} of a vector $v\in V$ is defined to be 
\[ \| v\| = \sqrt{\langle v, v \rangle}.\]

{\keya{\bf{Definition.}}} Given a vector space $V$ with an inner product, the {\keyb{\bf{distance}}}
between two vectors $v,w\in V$ is defined to be $\|v-w\|$.  

\endedxtext



\beginedxproblem{Calculate it 3}{\dpa3}

Let $v = \left[ \begin{array}{c} 2 \\ 1 \\ 0 \\ 4 \end{array} \right],$
and $w = \left[ \begin{array}{c} 4 \\ -7 \\ -3 \\ 6 \end{array} \right].$
What is the distance between $v$ and $w$?  

You can type sqrt(2) for $\sqrt{2}$.
  
\edXabox{type="numerical" expect="9" feqin="1"  tolerance=".01"}


\edXsolution{
We see that $v-w = \left[ \begin{array}{c} -2 \\ 8 \\ 3 \\ -2 \end{array} \right],$
so 
\[ \|v-w\| = \sqrt{(v-w)\dotprod (v-w)} = \sqrt{(-2)^2 + 8^2 + 3^2 + (-2)^2} = 9. \]
}

\endedxproblem

\beginedxproblem{Calculate it 4}{\dpa3}

Let $f(x) = 1$.  
Using the inner product on the vector space $C[0,2]$ as defined earlier, 
calculate $\|f\|$.  

\edXabox{type="numerical" expect="sqrt(2)" feqin="1"  tolerance=".01"}


\edXsolution{
\[ \|f\| = \sqrt{\langle f, f\rangle} = \sqrt{\int_0^2 f(x)^2 \ dx} = \sqrt{\int_0^2 1 \ dx} = \sqrt{2}.\]
}

\endedxproblem



\endedxvertical

\beginedxvertical{Orthogonality}

\doedxvideo{Orthogonality}{gKxxRZ2FtuY}

\beginedxtext{Orthogonal Definition}

{\keya{\bf{Definition.}}} Given a vector space $V$ with an inner product, we say two vectors $v,w\in V$ are {\keyb{\bf{orthogonal}}} if $\langle v,w\rangle = 0$.  

{\keya{\bf{Theorem.}}} Vectors $v$ and $w$ are orthogonal if and only if 
\[\|v\|^2 + \|w\|^2 = \|v+w\|^2.\]

\endedxtext


\endedxvertical

\beginedxvertical{Orthogonality Questions}

\beginedxproblem{Magnitudes}{\dpa1}

True or false: Given any two vectors $v,w$ in a vector space with an inner product, we have
\[ \|v+w\|^2 = \|v\|^2 + \|w\|^2.\]


\edXabox{expect="False" options="True","False"}

\edXsolution{
The Pythagorean Theorem holds only when $v$ and $w$ are orthogonal.  
}

\endedxproblem

\beginedxproblem{Orthogonal in R^3}{\dpa2}


Let $v = \left[ \begin{array}{c} -1 \\ 2 \\ 6 \end{array} \right]$
and $w = \left[ \begin{array}{c} 4 \\ -7 \\ a \end{array} \right]$ be vectors in $\R^3$.  
For what value of $a$ are $v$ and $w$ orthogonal to one another?  

\edXabox{type="numerical" expect="3" feqin="1"  tolerance=".01"}

\edXsolution{
We can calculate the inner product 
\[\langle v, w \rangle = -1(4) + 2(-7) + 6(a) = 6a - 18.\]
For $v$ and $w$ to be orthogonal, this inner product must equal zero.  This happens when $a=3$.  
}

\endedxproblem


\beginedxproblem{Orthogonal or not?}{\dpa2}

Let $f(x) = x$, $g(x) = 1$, and $h(x) = 1-x$.  
Using the inner product on the vector space $C[0,2]$ as described earlier, 
which pairs of functions are orthogonal?  Check all that are.    

\edXabox{type="oldmultichoice" expect="g and h" options="f and g","f and h","g and h"}

\edXsolution{
We can calculate $\langle f, g\rangle = 2$, $\langle f, h\rangle = -2/3$, and $\langle g,h \rangle = 0$.  
Hence the only pair which is orthogonal is $g$ and $h$.  
}

\endedxproblem

\endedxvertical




\beginedxvertical{Orthogonal Lists}

\doedxvideo{Orthogonal Lists}{pok4a_MtZgg}

\beginedxtext{Orthogonal and Orthonormal Lists}

{\keya{\bf{Definition.}}} Given a vector space $V$ with an inner product, we say a list $\{v_1; v_2; \ldots v_n\}$ in  $V$ is {\keyb{\bf{orthogonal}}} if $\langle v_i,v_j\rangle = 0$ for $i\ne j$.  
The list is {\keyb{\bf{orthonormal}}} if it is orthogonal, and $\|v_i\| = 1$ for all $i$.  


{\keya{\bf{Theorem.}}} Any orthogonal list of non-zero vectors is linearly independent.  

\endedxtext


\endedxvertical

\beginedxvertical{Orthogonal List Questions}


\beginedxproblem{List 1}{\dpa1}

True or false: If $v$ and $w$ are orthogonal, and $u$ is orthogonal to both $v$ and to $w$, then the 
list $\{u; v; w\}$ must be linearly independent.  

\edXabox{expect="False" options="True","False"}

\edXsolution{
The list is an orthogonal list, but one of the vectors could be $\veco$.  
}

\endedxproblem

\beginedxproblem{List 2}{\dpa1}

True or false: Every orthonormal list is linearly independent.  

\edXabox{expect="True" options="True","False"}

\edXsolution{
In an orthonormal list, all vectors have norm 1, so none are the zero vector.  Hence, by the theorem 
proven on the previous page, the list must be linearly independent.  
}

\endedxproblem

\endedxvertical





