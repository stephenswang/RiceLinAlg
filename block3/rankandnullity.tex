

\beginedxvertical{Page One}

\beginedxtext{Preliminaries}





At the end of this sequence, and after some practice, you should be able to:

\begin{itemize}
\item
\item

\item Determine the rank and nullity of a linear transformation $T: \R^n \rightarrow \R^m$, given its
standard matrix
\item Find bases for the image and kernel of  a linear transformation $T: \R^n \rightarrow \R^m$, given its standard matrix
\item Apply the Rank-Nullity Theorem.  
\end{itemize}


For time budgeting purposes, this sequence has 2 videos totaling 12 minutes, 
plus some questions.  




\endedxtext

\endedxvertical



\beginedxvertical{Rank and Nullity}


\doedxvideo{Defining Rank and Nullity}{uinL2McUGAA}


\beginedxtext{Definitions}


{\keya{\bf{Definitions.}}}  
Given a linear transformation $T: V\rightarrow W$, the {\keyb{\bf{nullity}}} of $T$ is the
dimension of the kernel of $T$, and the {\keyb{\bf{rank}}} of $T$ is the
dimension of the image of $T$. 

If a matrix $A$ is the standard matrix for a linear transformation $T:\R^n \rightarrow \R^m$, 
then the rank and nullity of $A$ are the same as the rank and nullity of $T$.  


\endedxtext


\endedxvertical



\beginedxvertical{Rank and Nullity Questions}


\beginedxproblem{Rank and Nullity 1}{\dpa1}

Let $T: \R^2 \rightarrow \R^2$ be the transformation that rotates vectors by $90^\circ$ counterclockwise.

What is the rank of $T$?  

\edXabox{type="numerical" expect="2"}


What is the nullity of $T$?  

\edXabox{type="numerical" expect="0"}


\edXsolution{
}

\endedxproblem




\beginedxproblem{Rank and Nullity 2}{\dpa1}

Let $T: \R^2 \rightarrow \R^2$ be the transformation that sends all vectors to $\veco$.  


What is the rank of $T$?  

\edXabox{type="numerical" expect="0"}


What is the nullity of $T$?  

\edXabox{type="numerical" expect="2"}


\edXsolution{
}

\endedxproblem




\beginedxproblem{Rank and Nullity 3}{\dpa1}

Suppose $\dim V = 5$.  Let $T: V\rightarrow V$ be a linear transformation.  If we know that $T$ is
not onto, what is the largest possible rank of $T$?  

\edXabox{type="numerical" expect="4"}


\edXsolution{
}

\endedxproblem


\endedxvertical


\beginedxvertical{Rank-Nullity Theorem}

\doedxvideo{Rank-Nullity Theorem}{la7BvLezg1I}


\beginedxtext{Rank-Nullity Theorem}

{\keya{\bf{Rank-Nullity Theorem.}}}  If $T:V\rightarrow W$ is a linear transformation and $\dim V = n < \infty$, then $\dim V - \mathrm{Nullity}(T) = \mathrm{Rank}(T)$.  




\begin{edXshowhide}{Proof of Theorem}

Suppose that $\dim V = n$, and that 
$\mathrm{Nullity}(T) = k$.  Then we can find a basis of $\mathrm{Ker}(T)$ consisting of $k$ vectors
$\{v_1; v_2; \ldots v_k\}$.  Since this is a linearly independent list of vectors inside $V$, we can extend
it to form a basis of $V$.  Since $\dim V = n$, this means adding $n-k$ vectors $u_1, u_2, \ldots u_{n-k}$.
So we have $\{v_1; v_2; \ldots v_k; u_1; u_2; \ldots u_{n-k} \}$ as a basis of $V$.  

For $i  = 1, 2, \ldots, n-k$, let $w_i = T(u_i)$.  We claim that 
the list $\{w_1; w_2; \ldots w_{n-k}\}$ is a basis of the image of $T$.  If so, then the dimension of the image
of $T$ -- i.e., the rank of $T$ -- would be
$n-k$, as desired.  

To show it forms a basis of the image of $T$, we first show that it is linearly independent.  
Suppose that a linear combination $b_1w_1 + b_2w_2 + \ldots b_{n-k}w_{n-k} = \veco$.  Then we have
\[ T(b_1u_1 + b_2u_2 + b_{n-k}u_{n-k}) = b_1T(u_1)  + b_2T(u_2) + \ldots + b_{n-k}T(u_{n-k}) = \veco,\]
so $b_1u_1 + b_2u_2 + b_{n-k}u_{n-k} \in \mathrm{Ker}(T).$  Since  
$\{v_1; v_2; \ldots v_k\}$ spans the kernel of $T$, this means that 
\[b_1u_1 + b_2u_2 + b_{n-k}u_{n-k} = a_1v_1 + a_2v_2 + \ldots +a_kv_k\]
for some scalars $a_i$; alternatively,
\[ -a_1v_1 - a_2v_2 - \ldots - a_kv_k + b_1u_1 + b_2u_2 + b_{n-k}u_{n-k} = \veco. \]
However, we know that $\{v_1; v_2; \ldots v_k; u_1; u_2; \ldots u_{n-k} \}$ is linearly independent,
so all of these coefficients must be zero.  In particular, all $b_i$ are zero.  This proves that $\{w_1; w_2; \ldots
w_{n-k} \}$ is linearly independent. 

To show that this list spans the image of $T$, take an arbitrary vector $T(x)$ in the image of $T$, where $x\in V$.
Since $\{v_1; v_2; \ldots v_k; u_1; u_2; \ldots u_{n-k} \}$ is a basis of $V$, we must be able to write
$x$ as a linear combination
\[x = a_1v_1 + \ldots + a_k v_k + b_1u_1 + \ldots + b_{n-k}u_{n-k}.\]
Hence 
\[
\begin{array}{rcl} T(x) &= & a_1T(v_1) + \ldots + a_kT(v_k) + b_1 T(u_1) + \ldots + b_{n-k}T(u_{n-k}) \\
& = & \veco + \ldots \veco + b_1w_1 + \ldots + b_{n-k}w_{n-k}.\end{array}
\]
This shows that $T(x)$ is a linear combination of the vectors $w_i$, so $\{w_1; w_2; \ldots
w_{n-k} \}$ spans the image of $T$.  Thus it is a basis of the image of $T$, and our proof is complete.  

\end{edXshowhide}



\endedxtext


\endedxvertical


\beginedxvertical{Examples}

\beginedxproblem{Rank-Nullity Question 1}{\dpa1}

Let $T: \R^9 \rightarrow \R^3$ be a linear transformation.  If the image of $T$ is a two-dimensional 
subspace of $\R^3$, what is the dimension of its kernel?

\edXabox{type="numerical" expect="7"}


\edXsolution{
}

\endedxproblem

\beginedxproblem{Rank-Nullity Question 2}{\dpa1}

Let $T: P_4 \rightarrow \R^7$ be a linear transformation.  What is the largest possible rank of $T$?

\edXabox{type="numerical" expect="5"}

Is is possible for $T$ to be onto?  

\edXabox{expect="No" options="Yes","No"}


\edXsolution{
}

\endedxproblem


\beginedxproblem{Rank-Nullity Question 3}{\dpa1}

Let $T: \R^4 \rightarrow P_6$ be a linear transformation which is not into.  What is the maximum possible
dimension of the image of $T$?    

\edXabox{type="numerical" expect="3"}


\edXsolution{
}

\endedxproblem


\beginedxproblem{Rank-Nullity Question 4}{\dpa1}

Suppose $V$ and $W$ are vector spaces with the same (finite) dimension $n$.  

If $T: V\rightarrow W$ is a linear transformation which is onto, what is the rank of $T$?  


\edXabox{type="formula" expect="n" samples="n@1:2#5" feqin="1" tolerance=".001"}

Therefore, what is the nullity of $T$?  

\edXabox{type="formula" expect="0" samples="n@1:2#5" feqin="1" tolerance=".001"}

Therefore, we can conclude that $T$ must be what?  (Answer yourself.)  

\edXsolution{
}

\endedxproblem


\beginedxproblem{Rank-Nullity Question 5}{\dpa1}

Suppose $V$ and $W$ are vector spaces with the same (finite) dimension $n$.  

If $T: V\rightarrow W$ is a linear transformation which is into, what is the nullity of $T$?  


\edXabox{type="formula" expect="0" samples="n@1:2#5" feqin="1" tolerance=".001"}

Therefore, what is the rank of $T$?  

\edXabox{type="formula" expect="n" samples="n@1:2#5" feqin="1" tolerance=".001"}

Therefore, we can conclude that $T$ must be what?  (Answer yourself.)  

\edXsolution{
}

\endedxproblem



\endedxvertical


\beginedxvertical{One Last Example}

\doedxvideo{Another Rank-Nullity Example}{EOJeHCQHxRw}


\beginedxproblem{Question}{\dpa1}

Suppose $\dim V = 14$ and let $W$ be a subspace of $V$. 

True or false: If we construct a linear transformation $T: V \rightarrow \R^2$ such that $W = \mathrm{Ker}(T)$,
then $\dim W$ must be 12.  


\edXabox{expect="False" options="True","False"}


\edXsolution{
}

\endedxproblem


\endedxvertical


\beginedxvertical{Nullity}



\doedxvideo{Calculating Nullity}{vKR7ciBTuCc}


\beginedxproblem{Example 1}{\dpa2}


Let $T: \R^{8}\rightarrow \R^{11}$ be a linear transformation with standard matrix $A$.  If 
there are six pivots when $A$ is row-reduced, what is the nullity of $T$?  

\edXabox{type="numerical" expect="2" feqin="1"}

\edXsolution{
We showed in the video that the nullity of $T$ equals the number of free variables in $A$.  $A$ has eight columns, and six pivots when row reduce, so there must be $8-6$, or 2, free variables.  
}

\endedxproblem

\endedxvertical



\beginedxvertical{Rank}


\doedxvideo{Calculating Rank}{rnePz12JGvk}


\beginedxproblem{Basis for Image}{\dpa1}

Suppose $T$ has a standard matrix $A$ which row reduces to the matrix 
\[
R = \left[\begin{array}{ccccc} 1 & 2 & 0 & 0 & -3 \\ 0 & 0 & 1 & 0 & -2 \\ 0 & 0 & 0 & 1  & 2 \\ 
0 & 0 & 0 & 0  & 0 
\end{array} \right].
\]

True or False: We can conclude that the list
\[ \left\{ \left[\begin{array}{c} 1 \\ 0 \\ 0 \\ 0
\end{array} \right];  
\left[\begin{array}{c} 0 \\ 1 \\ 0 \\ 0
\end{array} \right];  \left[\begin{array}{c} 0 \\ 0 \\ 1 \\ 0
\end{array} \right] \right\} \]
is a basis for the image of $T$.  



\edXabox{expect="False" options="True","False"}

\edXsolution{False.  The first, third, and fourth columns have pivots, but this means that the
first, third, and fourth columns of $A$ will form a basis for $\mathrm{Image}(T)$, not necessarily
the first, third, and fourth columns of $R$.}

\endedxproblem






\endedxvertical


\beginedxvertical{Summary}

\beginedxtext{Summary of Rank and Nullity for Linear Transformations from R^n to R^m}


Let $T: \R^{n}\rightarrow \R^{m}$ be a linear transformation with standard matrix $A$. 
The nullity of $T$ will equal the number of free variables when $A$ is row reduced, and the
rank of $T$ will equal the number of pivot columns when $A$ is row reduced.  The columns of $A$ which
correspond to the pivots will form a basis of the image of $T$.  


\endedxtext

\endedxvertical
