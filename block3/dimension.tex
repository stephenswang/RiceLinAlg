

\beginedxvertical{Page One}

\beginedxtext{Preliminaries}





At the end of this sequence, and after some practice, you should be able to:

\begin{itemize} 
\item Find coordinates of a vector relative to a basis.  
\item Dimension... 
\end{itemize}


For time budgeting purposes, this sequence has x videos totaling X minutes, 
plus some questions.  




\endedxtext

\endedxvertical








\beginedxvertical{Consequences of Having Coordinates}



\doedxvideo{Consequences of Having Coordinates}{sNoKjBJ9tpE}


\beginedxtext{Coordinatization as Linear Transformation}

{\keya{\bf{Proposition.}}}  If $\mathcal{B}  = \{v_1; v_2; \ldots ; v_n\}$ is a basis of $V$, then the
coordinatization map $C: V \rightarrow \R^n$ given by $C(w) = [w]_{\mathcal{B}}$ is a linear transformation
which is both into and onto.

\begin{edXshowhide}{Proof of Proposition}

To show that $C$ is a linear transformation...

In the video, we proved that $C$ is into and onto. 

\end{edXshowhide}



{\keya{\bf{Proposition.}}}  If $\mathcal{B}  = \{v_1; v_2; \ldots ; v_n\}$ is a basis of $V$, and $k>n$, then any list
of $k$ vectors in $V$ must be linearly dependent.  

\endedxtext


\endedxvertical




\beginedxvertical{Further Consequences}


\beginedxproblem{Two Bases? 1}{\dpa1}


Recall the proposition we just proved: 

{\keya{\bf{Proposition.}}}  If $\mathcal{B}  = \{v_1; v_2; \ldots ; v_n\}$ is a basis of $V$, and $k>n$, then any list
of $k$ vectors in $V$ must be linearly dependent.  

Now suppose that $\mathcal{B}$ is a basis of $V$ with 5 vectors, and let $\mathcal{C}$ be a list of 7 vectors
in $V$.  Is it possible for $\mathcal{C}$ to be a basis of $V$?  


\edXabox{expect="No" options="Yes","No"}

\edXsolution{No.  Since $V$ has a basis which contains 5 vectors, the list $C$, which contains 7 vectors, must be linearly dependent.  By definition, a linearly dependent list cannot form a basis.
}

\endedxproblem


\beginedxproblem{Two Bases? 2}{\dpa1}

Now suppose that $\mathcal{C}$ is a basis of $V$ with 7 vectors.  Is it possible for there to be a basis $\mathcal{B}$ of $V$ with 5 vectors?  


\edXabox{expect="No" options="Yes","No"}

\edXsolution{If $B$ is a list containing 5 vectors and forms a basis for $V$, then the list $C$, which contains 7 vectors, must be linearly dependent.  But, by definition, a linearly dependent list cannot be a basis.  This contradicts the assumption that $C$ formed a basis for $V$.
}

\endedxproblem

\endedxvertical




\beginedxvertical{Dimension}



\doedxvideo{Dimension}{gKmyYkYB0Qo}

\beginedxtext{Dimension Definition}

{\keya{\bf{Proposition.}}}  All bases of a vector space $V$ must have the same number of vectors.  
  

{\keya{\bf{Definition.}}}  
The {\keyb{\bf{dimension}}} of a vector space $V$ is defined to be the number of vectors in 
a basis of $V$.  If $V$ has no finite basis, we say that $V$ is infinite-dimensional.  


\endedxtext


\beginedxproblem{Dimension Question 1}{\dpa1}

True or False: If the dimension of $V$ is 4, then every list of 4 vectors in $V$ must be a basis of $V$.  

\edXabox{expect="False" options="True","False"}

\edXsolution{False.  For example, $\R^4$ has basis consisting of the standard unit vectors $\{e_1, e_2, e_3, e_4\}$, and thus $\R^4$ is dimension 4.  The list $\{e_1, 2e_1, 3e_1, 4e_1\}$ contains 4 vectors, but they are all linearly dependent.  This list spans a one-dimensional subspace of $\R^4$, and thus is not a basis of $\R^4$.
}

\endedxproblem



\beginedxproblem{Dimension Question 2}{\dpa1}

True or False: If the dimension of $V$ is 6, then there must exist a basis of $V$ with exactly 6 vectors.

\edXabox{expect="True" options="True","False"}

\edXsolution{We are assuming that $V$ has dimension 6.  In order to define the dimension of $V$, there must be a basis of $V$ containing 6 vectors.
}

\endedxproblem

\beginedxproblem{Dimension Question 3}{\dpa1}

Let $V = \R^2$.  What is the dimension of $V$?  

\edXabox{type="numerical" expect="2"}

\edXsolution{The list of standard vectors $$\{e_1, e_2\}=\left\{\left[ \begin{array}{c} 1 \\ 0 \end{array} \right], \left[ \begin{array}{c} 0 \\ 1 \end{array} \right]\right\}$$ forms a basis for $\R^2$.  Thus $\R^2$ is 2-dimensional. 
}

\endedxproblem


\beginedxproblem{Dimension Question 4}{\dpa1}


\begin{center}
\includesvg[450]{c3s2dimension}
\end{center}

Let $W$ be the subspace of $\R^2$ drawn above.  What is the dimension of $W$?  


\edXabox{type="numerical" expect="1"}

\edXsolution{If we take $v$ to be a vector of any length that lies in the direction of $W$, then the set of linear combinations of $v$ will be all of $W$.  Then, $\{v\}$ is a linearly independent list that spans $W$, and thus $\{v\}$ is a basis of $W$.  By definition, the dimension of $W$ is the number of vectors in any basis for $W$.  This is 1.
}

\endedxproblem


\endedxvertical




\beginedxvertical{Dimension and Subspaces}



\doedxvideo{Dimension and Subspaces}{5mlxz2ysoEg}


\beginedxtext{Dimension Propositions}

{\keya{\bf{Proposition (A).}}}  If $\dim V = n < \infty$, and $W$ is any subspace of $V$, then any
linearly independent list of vectors in $W$ can be extended to form a basis of $W$.  

{\keya{\bf{Proposition.}}} If $\dim V = n$, then any linearly independent list of vectors in $V$ has
at most $n$ vectors.  Any linearly independent list of vectors in $V$ that has exactly $n$ vectors must
be a basis of $V$.  


{\keya{\bf{Proposition.}}} If $\dim V = n$, then any subspace $W$ of $V$ has dimension at most $n$.  
If $\dim W = n$, then $W = V$.  


{\keya{\bf{Proposition (B).}}} Any finite list of vectors that spans a vector space $V$ contains a basis of $V$.  

{\keya{\bf{Proposition.}}} If $\dim V = n$, then any list of vectors that spans $V$ has
at least $n$ vectors.  Any  list of vectors that spans $V$ that has exactly $n$ vectors must
be a basis of $V$.  


\begin{edXshowhide}{Proof of Proposition (A)}

stuff

\end{edXshowhide}


\begin{edXshowhide}{Proof of Proposition (B)}

stuff

\end{edXshowhide}



\endedxtext

\endedxvertical
