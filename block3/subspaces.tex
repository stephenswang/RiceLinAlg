

\beginedxvertical{Page One}

\beginedxtext{Preliminaries}





At the end of this sequence, and after some practice, you should be able to:

\begin{itemize}
\item Identify subsets of a vector space which are subspaces.
\item Prove that some abstractly defined subsets of vector spaces are subspaces.

\end{itemize}


For time budgeting purposes, this sequence has 3 videos totaling 18 minutes, 
plus some questions.  




\endedxtext

\endedxvertical



\beginedxvertical{Subspaces}


\doedxvideo{Subspaces}{Q0kn88woMvU}


\beginedxproblem{Subset? Subspace?}{\dpa1}

Is $\R^4$ a subset of $\R^5$?  


\edXabox{expect="No" options="Yes","No"}

Is $\R^4$ a subspace of $\R^5$?  


\edXabox{expect="No" options="Yes","No"}

\edXsolution{Elements of $\R^4$ are column vectors with 4 entries, whereas elements of $\R^5$ have 5 entries. Thus, an element of $\R^4$ is not an element of $\R^5$, so $\R^5$ is not a subset of, and thus not a subspace of, $\R^5$.
}

\endedxproblem

\endedxvertical



\beginedxvertical{Criteria for being a Subspace}

\doedxvideo{Subspace Criteria}{pQWBhzg_6Bc}

\beginedxtext{Subspaces}

{\keya{\bf{Definition.}}}  
Suppose $V$ is a vector space over a field $F$.  A subset $W \subset V$ is a {\keyb{\bf{subspace}}}  
of $V$ if $W$ is a vector space over $F$ in its own right, using the same vector operations as in $V$.  

{\keya{\bf{Proposition.}}}  
A subset $W$ of a vector space $V$ is a subspace of $V$ if the following conditions hold:
\begin{itemize}
\item
$W$ is closed under addition; that is, for all vectors $w_1,w_2 \in W$, the sum $w_1+w_2$ is also an
element of $W$.  
\item
$W$ is closed under scalar multiplication; that is, for any vector $w\in W$ and scalar $a \in F$, the
product $aw$ is also an element of $W$.  
\item
$W$ contains $\veco$.  
\end{itemize}




\endedxtext


\endedxvertical



\beginedxvertical{Subspace Examples}



\beginedxproblem{First Quadrant}{\dpa1}

Let $W$ be the set of vectors in $\R^2$ which lie in (or on the boundary of) the first quadrant.  
That is,
\[ W = \left\{ \left[ \begin{array}{c} x_1 \\ x_2 \end{array} \right] : x_1, x_2 \ge 0\right\}. \]



\begin{center}
\includesvg[300]{c3s1subspace1}   
\end{center}

Is $W$ closed under addition?

\edXabox{expect="Yes" options="Yes","No"}

Is $W$ closed under scalar multiplication?

\edXabox{expect="No" options="Yes","No"}

Does $W$ contain the zero vector?

\edXabox{expect="Yes" options="Yes","No"}

Is $W$ a subspace of $\R^2$?  

\edXabox{expect="No" options="Yes","No"}


\edXsolution{ $W$ contains $\veco$ and is closed under vector addition, but if you scale
a vector such as $\left[ \begin{array}{c} 1 \\ 1 \end{array} \right]$ in $W$ by, say, $-2$, the result
is not in $W$.  Thus $W$ is not closed under scalar multiplication and is not a subspace of $\R^2$.  
 }
 
\endedxproblem


\beginedxproblem{First and Third Quadrant}{\dpa1}

Let $U$ be the set of vectors in $\R^2$ which lie in (or on the boundary of) the first quadrant or
the third quadrant.  
That is,
\[ U = \left\{ \left[ \begin{array}{c} x_1 \\ x_2 \end{array} \right] : x_1, x_2 \ge 0 \ \mathrm{or} \ x_1,x_2 \le 0\right\}. \]

\begin{center}
\includesvg[300]{c3s1subspace2}   
\end{center}

Is $U$ closed under addition?

\edXabox{expect="No" options="Yes","No"}

Is $U$ closed under scalar multiplication?

\edXabox{expect="Yes" options="Yes","No"}

Does $U$ contain the zero vector?

\edXabox{expect="Yes" options="Yes","No"}

Is $U$ a subspace of $\R^2$?  

\edXabox{expect="No" options="Yes","No"}


\edXsolution{ $U$ contains $\veco$ and is closed under scalar multiplication, but if you add
$\left[ \begin{array}{c} 3 \\ 1 \end{array} \right]$ to 
$\left[ \begin{array}{c} -1 \\ -3 \end{array} \right]$, the result
is not in $U$.  Thus $U$ is not closed under addition and is not a subspace of $\R^2$.  
 }
 
\endedxproblem


\beginedxproblem{Diagonal Line}{\dpa1}

Let $Y$ be the set of vectors in $\R^2$ which lie on the line $x_1 = x_2$
That is,
\[ Y = \left\{ \left[ \begin{array}{c} c \\ c \end{array} \right] : c \in \R \right\}. \]

\begin{center}
\includesvg[300]{c3s1subspace3}   
\end{center}

Is $Y$ closed under addition?

\edXabox{expect="Yes" options="Yes","No"}

Is $Y$ closed under scalar multiplication?

\edXabox{expect="Yes" options="Yes","No"}

Does $Y$ contain the zero vector?

\edXabox{expect="Yes" options="Yes","No"}

Is $Y$ a subspace of $\R^2$?  

\edXabox{expect="Yes" options="Yes","No"}


\edXsolution{ If you add or scale vectors in $Y$, the result is still in $Y$.  $Y$ also contains
the zero vector.  Thus $Y$ is a subspace of $\R^2$.  
 }
 
\endedxproblem

\beginedxproblem{Horizontal Plane}{\dpa1}

Let $Z$ be the set of vectors in $\R^3$ whose third coordinate is -1.  
That is,
\[ Z = \left\{ \left[ \begin{array}{c} x_1 \\ x_2 \\ -1 \end{array} \right] : x_1, x_2 \in \R \right\}. \]

\begin{center}
\includesvg[300]{c3s1subspace5}   
\end{center}

Is $Z$ closed under addition?

\edXabox{expect="No" options="Yes","No"}

Is $Z$ closed under scalar multiplication?

\edXabox{expect="No" options="Yes","No"}

Does $Z$ contain the zero vector?

\edXabox{expect="No" options="Yes","No"}

Is $Z$ a subspace of $\R^2$?  

\edXabox{expect="No" options="Yes","No"}


\edXsolution{ 
This is one where you have to remember that the vectors contained in $Z$ are not the arrows that lie {\emph{"along"}} the pictured set, but rather are the points within the set (or the arrows that are based at the
origin and end in the set).  For instance,  $\left[ \begin{array}{c} 2 \\ 0 \\ -1 \end{array} \right]$ and 
$\left[ \begin{array}{c} 0 \\ 3 \\ -1 \end{array} \right]$ are elements of $Z$, but their sum is not.  
 }
 
\endedxproblem


\endedxvertical



\beginedxvertical{Span}


\doedxvideo{Span as Subspace}{hjeYq1Ly0BE}





\endedxvertical



\beginedxvertical{Other Important Subspaces}



\beginedxproblem{Kernel}{\dpa1}

Suppose that $T: V\rightarrow W$ is a linear transformation.  


Does the kernel of $T$ contain the zero vector of $V$?  

\edXabox{expect="Yes" options="Yes","No"}

Prove (on your own) that the kernel of $T$ is closed under addition.  

\edXabox{expect="I have proven it" options="I have proven it"}

Prove (on your own) that the kernel of $T$ is closed under scalar multiplication.  

\edXabox{expect="I have proven it" options="I have proven it"}


Is the kernel of $T$ a subspace of $V$?  

\edXabox{expect="Yes" options="Yes","No"}


\edXsolution{ We know $T(\veco) = \veco$, so $\veco \in \mathrm{Ker}(T).$  

If $v_1, v_2 \in \mathrm{Ker}(T)$, then $T(v_1) = T(v_2) = \veco$.  
Therefore, \[T(v_1+v_2) = T(v_1) + T(v_2) = \veco + \veco = \veco.\]
We conclude that $v_1+v_2 \in \mathrm{Ker}(T),$ so the kernel is closed under addition.  

Similarly, for any scalar $a$, 
\[T(av_1) = aT(v_1)= a\veco = \veco,\]
so the kernel is closed under scalar multiplication.  

Hence it is a subspace of the domain.  
 }
 
\endedxproblem



\beginedxtext{Image}

Similarly, try on your own to show that the image of $T$ is a subspace of $W$.  

\endedxtext


\endedxvertical



\beginedxvertical{Review Question}

\beginedxproblem{Number of Subspaces}{\dpa1}

True or false: Every non-zero vector space has at least two subspaces.  

\edXabox{expect="True" options="True","False"}

\edXsolution{ If $V$ is a vector space, then $\{\veco\}$ and $V$ are both subspaces
of $V$.  If $V$ is non-zero, then these two subspaces are different from one another.  
}
 
\endedxproblem



\endedxvertical
