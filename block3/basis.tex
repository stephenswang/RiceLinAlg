

\beginedxvertical{Page One}

\beginedxtext{Preliminaries}





At the end of this sequence, and after some practice, you should be able to:

\begin{itemize}
\item Determine when a list of vectors in $\R^n$ is a basis of $\R^n$.    
\item Find coordinates of a vector relative to a basis.  
\end{itemize}


For time budgeting purposes, this sequence has x videos totaling X minutes, 
plus some questions.  




\endedxtext

\endedxvertical



\beginedxvertical{Basis}

\doedxvideo{Definition of Basis}{cmx119UrFwg}


\beginedxtext{Basis Definition}


{\keya{\bf{Definition.}}}  
Suppose $V$ is a vector space over a field $F$.  A list of vectors $\{v_1; v_2; \ldots v_n\}$ in $V$ is a 
{\keyb{\bf{basis}}}  
of $V$ if the list is linearly independent and spans $V$.  

{\keya{\bf{Definition.}}}  
The list $\{e_1; e_2; \ldots e_n\}$ in $\R^n$ is called the {\keyb{\bf{standard basis}}} of $\R^n$.  




\endedxtext

\endedxvertical



\beginedxvertical{Basis Questions}

\beginedxproblem{Is it a basis? 1}{\dpa1}


Let $v_1 = \left[ \begin{array}{c} 1 \\ 0 \\1 \end{array} \right]$, 
$v_2 = \left[ \begin{array}{c} 1 \\ 1 \\ 0 \end{array} \right]$, and 
$v_3 = \left[ \begin{array}{c} 2 \\ -1 \\ 4 \end{array} \right]$ be three vectors in $\R^3$.  

Is the list $\{v_1; v_2; v_3\}$ linearly independent?  (You may need to do some row reduction.)  

\edXabox{expect="Yes" options="Yes","No"}

Does the list $\{v_1; v_2; v_3\}$ span $\R^3$?  

\edXabox{expect="Yes" options="Yes","No"}

Is the list  $\{v_1; v_2; v_3\}$ a basis of $\R^3$?  

\edXabox{expect="Yes" options="Yes","No"}

\edXsolution{
We can answer these questions by row reducing the matrix with columns $v_1$, $v_2$, and $v_3$.  If we
do this row reduction, we obtain the $3\times 3$ identity matrix.  This has a pivot in every column (i.e., no free variables),
so the list $\{v_1; v_2; v_3\}$ is linearly independent.  In addition, it has a pivot in every row, so the 
list spans $\R^3$.  Since both conditions hold, the list is a basis of $\R^3$.  
}

\endedxproblem



\beginedxproblem{Is it a basis? 2}{\dpa1}


Now consider the matrix $A = \left[ \begin{array}{ccc} 1 & 1 & 2 \\ 0 & 1 & -1 \\ 1 & 0 & 4 \end{array}\right].$  

$A$ is a vector in the vector space of $3\times 3$ matrices $M_{3\times 3}$.  


Is the list $\{A\}$ linearly independent in 
$M_{3\times 3}$?

\edXabox{expect="Yes" options="Yes","No"}

Does $\{A\}$ span $M_{3\times 3}$? 

\edXabox{expect="No" options="Yes","No"}

Is the list $\{A\}$ a basis of  $M_{3\times 3}$? 

\edXabox{expect="No" options="Yes","No"}

\edXsolution{
A list which only has one non-zero vector in it is automatically linearly independent.  
However, $\{A\}$ does not span the space $M_{3\times 3}$, since there are many matrices that cannot
be expressed as $cA$ for some scalar $c$.  Hence the list is not a basis of $M_{3\times 3}$.  
}

\endedxproblem

\endedxvertical



\beginedxvertical{Bases in R^n}



\beginedxproblem{Short list}{\dpa1}

Let's go back to $\R^n$ and recall a few facts.  

If a list of vectors in $\R^n$ has $k$ vectors with $k<n$, is it possible for the 
list to be linearly independent?

\edXabox{expect="Yes" options="Yes","No"}

Is it possible for the list to span $\R^n$?  

\edXabox{expect="No" options="Yes","No"}
 
\edXsolution{
We have shown in previous sections that $k$ vectors in $\R^n$ cannot span $\R^n$ if $k<n$.  
}

\endedxproblem

\beginedxproblem{Long list}{\dpa1}

If a list of vectors in $\R^n$ has $k$ vectors with $k>n$, is it possible for the 
list to be linearly independent?

\edXabox{expect="No" options="Yes","No"}

Is it possible for the list to span $\R^n$?  

\edXabox{expect="Yes" options="Yes","No"}
 
\edXsolution{
We have shown in previous sections that $k$ vectors in $\R^n$ cannot be linearly independent if $k>n$.  
}

\endedxproblem


\endedxvertical



\beginedxvertical{Finding Bases in R^n}

\doedxvideo{Bases in R^n}{NHgCtsjsArY}



\beginedxtext{Bases of R^n}

{\keya{\bf{Proposition.}}}  Every basis of $\R^n$ has exactly $n$ vectors.  

\endedxtext

\endedxvertical



\beginedxvertical{More Exercises}



\beginedxproblem{Basis? 3}{\dpa1}

Consider the vector $v$ in $\R^2$ shown below.  

\begin{center}
\includesvg[450]{c3s2basis1}
\end{center}

Which of the following statements is true of the list $L = \{v\}$?  


\edXabox{type="multichoice" expect="$L$ is a basis of some subspace of $\R^2$, but not of $\R^2$ itself" options="$L$ is a basis of $\R^2$","$L$ is a basis of some subspace of $\R^2$, but not of $\R^2$ itself","$L$ is not a basis of any subspace of $\R^2$"}

\edXsolution{
  
}

\endedxproblem


\beginedxproblem{Basis? 4}{\dpa1}

Consider the vectors $u,v,w$ in $\R^2$ shown below.  

\begin{center}
\includesvg[450]{c3s2basis3}
\end{center}

Which of the following statements is true of the list $L = \{u; v; w\}$?  


\edXabox{type="multichoice" expect="$L$ is not a basis of any subspace of $\R^2$" options="$L$ is a basis of $\R^2$","$L$ is a basis of some subspace of $\R^2$, but not of $\R^2$ itself","$L$ is not a basis of any subspace of $\R^2$"}

\edXsolution{
  
}

\endedxproblem

\beginedxproblem{Basis? 5}{\dpa1}

Consider the vectors $v,w$ in $\R^2$ shown below.  

\begin{center}
\includesvg[450]{c3s2basis2}
\end{center}

Which of the following statements is true of the list $L = \{v; w\}$?  


\edXabox{type="multichoice" expect="$L$ is a basis of some subspace of $\R^2$, but not of $\R^2$ itself" options="$L$ is a basis of $\R^2$","$L$ is a basis of some subspace of $\R^2$, but not of $\R^2$ itself","$L$ is not a basis of any subspace of $\R^2$"}

\edXsolution{
  
}

\endedxproblem


\beginedxproblem{Basis? 6}{\dpa1}

Consider the vectors $v,w$ in $\R^2$ shown below.  

\begin{center}
\includesvg[450]{c3s2basis4}
\end{center}

Which of the following statements is true of the list $L = \{v; w\}$?  


\edXabox{type="multichoice" expect="$L$ is not a basis of any subspace of $\R^2$" options="$L$ is a basis of $\R^2$","$L$ is a basis of some subspace of $\R^2$, but not of $\R^2$ itself","$L$ is not a basis of any subspace of $\R^2$"}

\edXsolution{
  
}

\endedxproblem


\endedxvertical



\beginedxvertical{Coordinates}



\doedxvideo{Coordinates}{_m8lE_c9L3I}


\beginedxproblem{Where does it live?}{\dpa1}

Suppose $\mathcal{B} = \{v_1; v_2; \ldots v_9\}$ is a basis of a vector space $V$ (over the field $\R$).  Let $w \in V$.  Then  $[w]_{\mathcal{B}}$ is an element of what space?  


\edXabox{type="multichoice" expect="$\R^9$" options="$V$","$\R$","$\R^9$","None of the above"}

\edXsolution{
$[w]_{\mathcal{B}}$ is defined to be a column of nine scalars; specifically, the nine scalars $a_1, a_2, \ldots a_9$ such that $w = a_1 v_1 + \ldots a_9v_9$.  
}

\endedxproblem

\endedxvertical



\beginedxvertical{Coordinates Defined}

\beginedxtext{Subspaces}

{\keya{\bf{Proposition.}}}  If $\{v_1; v_2; \ldots ; v_n\}$ is a basis of $V$, then every $w\in V$
can be written as a linear combination of $v_1, v_2, \ldots v_n$ in exactly one way.  


{\keya{\bf{Definition.}}} If  $\mathcal{B} = \{v_1; v_2; \ldots ; v_n\}$ is a basis of a vector space $V$ over the field $F$, and 
$w\in V$ can be written as $w = a_1v_1 + a_2v_2 + \ldots + a_n v_n$ for scalars $a_i$, then 
the {\keyb{\bf{coordinates of $w$ relative to $\mathcal{B}$}}} is the vector $[w]_{\mathcal{B}} \in F^n$
given by 
\[ [w]_{\mathcal{B}} = \left[ \begin{array}{c} a_1 \\ a_2 \\ \vdots \\ a_n \end{array} \right]. \]

\endedxtext




\endedxvertical



\beginedxvertical{Some Examples}

\beginedxproblem{Example 1}{\dpa3}

Suppose $\mathcal{B} = \{v_1; v_2; v_3\}$ is a basis of a vector space $V$ (over $\R$).  

Let $w = 3v_3 - 2v_2$.  What is $[w]_{\mathcal{B}}$?  

\input{vectorentry.tex}


\edXabox{type="custom" cfn="VectorEntry" expect="[[0],[-2],[3]]"}


\edXsolution{
We can write $w$ as a linear combination of the basis vectors in order:
$w = 0 v_1 + (-2)v_2 + 3v_3$.  Hence $[w]_{\mathcal{B}} =\left[\begin{array}{c} 0 \\ -2  \\ 3 \end{array} \right]$.  

}

\endedxproblem



\beginedxproblem{Example 2}{\dpa3}

Let $p_1 = 1+t$, $p_2 = 2+t^2$, and $p_3 = t-t^2$.  Fact: $\mathcal{B} = \{p_1; p_2; p_3\}$ is a basis of  $\mathbb{P}_2$.  

What polynomial $q$ has the property that  $[q]_{\mathcal{B}} = \left[\begin{array}{c} 1 \\ 2  \\ 3 \end{array} \right]$?

Remember to type * for multiplication (so to enter $3t^2$, type 3*t^2).  

\edXabox{type="formula" expect="5+4*t-t^2" samples="t@1:10#10" tolerance=".001" feqin="1"}

\edXsolution{
If $[q]_{\mathcal{B}} = \left[\begin{array}{c} 1 \\ 2  \\ 3 \end{array} \right]$,
then $q = 1p_1 + 2p_2 + 3p_3.$  Thus we compute $q$ to be $5+4t -t^2$.  

}

\endedxproblem


\endedxvertical



\beginedxvertical{More Coordinates}


\doedxvideo{More Coordinates}{8v4ocK_SsL8}


\beginedxproblem{Where does it live? 2}{\dpa1}

Let $W$ be a subspace of $\R^9$.  
Suppose $\mathcal{B} = \{v_1; v_2; \ldots v_5\}$ is a basis of $W$.  Let $w \in W$.  Then  $[w]_{\mathcal{B}}$ is an element of what space?  


\edXabox{type="multichoice" expect="$\R^5$" options="$W$","$\R$","$\R^5$","$\R^9$","None of the above"}

\edXsolution{
$[w]_{\mathcal{B}}$ is defined to be a column of five scalars; specifically, the five scalars $a_1, a_2, \ldots a_5$ such that $w = a_1 v_1 + \ldots a_5v_5$.  
}

\endedxproblem


\beginedxproblem{Example 3}{\dpa3}


\begin{center}
\includesvg[450]{c3s2coords}
\end{center}

Let $\mathcal{B} = \{v_1; v_2\}$ as pictured; this is a basis of $\R^2$.  

Given the $w$ in the image, what is $[w]_{\mathcal{B}}$?  

\input{vectorentry2.tex}


\edXabox{type="custom" cfn="VectorEntry" expect="[[-1],[2]]"}


\edXsolution{
We can write $w$ as a linear combination of the basis vectors in order:
$w = -1 v_1 + 2v_2$.  Hence $[w]_{\mathcal{B}} =\left[\begin{array}{c}  -1  \\ 2 \end{array} \right]$.  

}

\endedxproblem

\beginedxproblem{Example 4}{\dpa3}

Let $v_1=\left[\begin{array}{c} 1 \\ 2  \\ 3 \end{array} \right]$ and $v_2 = \left[\begin{array}{c} 3 \\ 0 \\ 1 \end{array} \right]$.  Then $\mathcal{B} = \{v_1; v_2\}$ is a basis for some subspace $W$
in $\R^3$.  

What vector $w$ has the property that  $[w]_{\mathcal{B}} = \left[\begin{array}{c} 1 \\ 2 \end{array} \right]$?

\input{vectorentry2.tex}


\edXabox{type="custom" cfn="VectorEntry" expect="[[7],[2],[5]]"}

\edXsolution{
If $[w]_{\mathcal{B}} = \left[\begin{array}{c} 1 \\ 2  \end{array} \right]$,
then $w = 1v_1 + 2v_2.$  Thus we compute $q$ to be $\left[\begin{array}{c} 7 \\ 2  \\ 5 \end{array} \right].$ 
}

\endedxproblem


\beginedxproblem{Example 5}{\dpa3}

Use the same $\mathcal{B}$ and $W$ as in the previous problem.  The vector 
$z = \left[\begin{array}{c} -3 \\ 6  \\ 7 \end{array} \right]$ is an element of $W$.  

What is $[z]_{\mathcal{B}}$?

\input{vectorentry2.tex}


\edXabox{type="custom" cfn="VectorEntry" expect="[[3],[-2]]"}

\edXsolution{
We can find $z$ as a linear combination of $v_1$ and $v_2$ by row reduction.  
We get that $z = 3v_1 - 2v_2$, so $[z]_{\mathcal{B}} = \left[\begin{array}{c} 3 \\ -2 \end{array} \right]$.
}

\endedxproblem


\endedxvertical

