

\beginedxvertical{Linear Algebra on EdX}


\beginedxtext{Welcome}


Welcome to Rice Math 355.1x: Linear Algebra, Part 1!  



Linear algebra is at the core of all of modern mathematics, and is used everywhere from statistics and data science, to economics, physics and electrical engineering.  
In this 8-week course, you'll learn about topics such as: the relationships between linear equations, matrices, and linear transformations; the principles of vector and matrix operations; the significance of basis and dimension of a vector space; and the applications of inner products and orthogonality. 

This covers about half of the material of a typical semester-long linear algebra course.  Stay tuned for Part 2, which
will discuss determinants, eigenvectors and eigenvalues, and the Spectral Theorem and Singular Value Decomposition!  

\endedxtext


\doedxvideo{Course Trailer}{hcNrbnhj71w}


\endedxvertical



\beginedxvertical{Course Staff}

\beginedxtext{Meet the Professor}

The instructor of this course is Stephen Wang, an Associate Teaching Professor at Rice University in Houston, Texas, USA.  
You can get to know Steve better by watching the following video.  

\endedxtext

\doedxvideo{About Steve}{hcNrbnhj71w}

\beginedxtext{Other Contributors}

Will Stagner and Xingya Wang (no relation to Steve!) are our course assistants.  You'll run into them as they answer questions in the discussion forum.  They are both graduate students in mathematics at Rice University.  

Tam Cheetham-West, Alexis Johnson, Alex Teich, and Wei Wu also helped to create and edit course content.  

This course is supported by a grant from Rice Online, which also provided technical assistance, particularly from 
Mart\'{i}n Calvi, Rhonda Humbird, and Seth Tyger.  It was primarily built using \href{https://people.csail.mit.edu/ichuang/edx/latex2edx.php}{latex2edx}.  

\endedxtext


\endedxvertical



\beginedxvertical{Course Requirements}


\beginedxtext{Prerequisites}

\mysection{Prerequisites}

This is listed as an intermediate-level course.  This is not because there are many mathematical prerequisites.$^*$  Rather, it is because this is primarily {\keyb{\bf{not}}} a computational course (although you will learn
how to do some computations); 
learning this subject is not principally about acquiring computational ability, but is more a matter of fluency in its language and theory.  You will have to think conceptually and theoretically about how all of these ideas fit together!  


($^*$: Every so often there will be some videos or questions which require some basic single-variable calculus, since there are  
nice examples and applications that use it.  However, it will be possible to pass this course with no knowledge of calculus whatsoever.  
Other than that, the only mathematical prerequisite is high-school algebra.)



\mysection{Honor Code}

Students are expected to abide by the EdX Honor Code Pledge.

By enrolling in an edX course or program, I agree that I will:

\begin{itemize}
\item
Complete all tests and assignments on my own, unless collaboration on an assignment is explicitly permitted.
\item
Maintain only one user account and not let anyone else use my username and/or password.
\item
Not engage in any activity that would dishonestly improve my results, or improve or hurt the results of others.
\item
Not post answers to problems that are being used to assess learner performance.
\end{itemize}

Students also may not use outside resources (looking up answers outside of the EdX course materials, using calculator or 
computer assistance) unless explicitly allowed.  Initially, no computer/calculator assistance outside of basic arithmetic functions
is allowed.  Later on, this will be relaxed, and students will receive notice that certain additional tools are allowable.  


\endedxtext

\endedxvertical


\beginedxvertical{Course Outline}


\beginedxtext{Outline}


\mysection{Course Outline}

This course is divided into four chapters.  Each chapter is divided into several sections (or learning sequences), which have questions and 
exercises embedded within them to help learners review and make connections.  

For students enrolled in the Verified Certificate program, each chapter also has a homework set attached to it, and there will be a final exam at the end.  Students will need to earn at least 65\% on the graded elements (the learning sequence exercises, the homework, and the final) to earn the
Verified Certificate.  

We expect that students will spend 4-6 hours per week on the course, if on the Certificate track.  Of course, the more
you put into the course, the more you'll get out of it!  



\mysection{Calendar}

\[
\begin{array}{|c|c|c|}
\hline
\mathrm{Content} & \mathrm{Opens} & \mathrm{Due} \\
\hline \hline
\mathrm{Chapter\ 1\ Learning\ Sequences} & \mathrm{Now} &  \mathrm{October \ 13 \ or \ 20}  \\
\mathrm{Chapter\ 1\ Homework} & \mathrm{Now} &  \mathrm{October \  20} \\
\hline
\mathrm{Chapter\ 2\ Learning\ Sequences} & \mathrm{October \ 7}  &   \mathrm{October \ 27}   \\
\mathrm{Chapter\ 2\ Homework} & \mathrm{October \ 7} &  \mathrm{October \ 27} \\
\hline
\mathrm{Chapter\ 3\ Learning\ Sequences} & \mathrm{October \ 14} &   \mathrm{November \  4 \ or \ 11}   \\
\mathrm{Chapter\ 3\ Homework} & \mathrm{October \ 14} & \mathrm{November \ 11}  \\
\hline
\mathrm{Chapter\ 4\ Learning\ Sequences} & \mathrm{October \ 28} & \mathrm{November \ 18}   \\
\mathrm{Chapter\ 4\ Homework} & \mathrm{October \ 28} & \mathrm{November \ 18} \\
\hline
\mathrm{Final\ Exam} & \mathrm{November  \ 5} & \mathrm{November  \ 25}  \\
\hline \hline \\
 & & 
\end{array}
\]
 
\endedxtext


\endedxvertical

\beginedxvertical{Discussion Forum}

\beginedxtext{Discussion Forum}

The Discussion Forum allows you to engage with course staff, and also with other students. This provides us with a unique opportunity to interact and share ideas with people from different backgrounds, different experiences, different cultures and different languages. In order for us to take advantage of this wealth of knowledge and viewpoints, please consider the following when you post:

\begin{itemize}
\item
Participate! You will get out what you put in, so be active.
\item
Have faith that course participants are acting with best intentions and have the best intentions when you post, too!

\item
Before posting, search the Discussion for similar comments. You can respond, elaborate or rebut the comment already posted! Discussions are more fun when you can follow a thread.

\item
Use the established forum categories. Making your own categories ends up confusing the conversation. Please search before asking.

\item
If you agree with a post or find it helpful, please upvote it. If you disagree with a post, respond using evidence and reasoning. Refer to the first guideline and please be respectful.

\item
Use your own words! If you include a quote or reference, when possible also provide a citation (book, URL, etc).

\item
Use correct grammar and spell-check. Also, please do not use ALL CAPS. We don't like being yelled at!

\item
Slang words and abbreviations vary across cultures, so please avoid these as much as possible.

\end{itemize}

You'll get an opportunity to use the discussion forum on the next page!  

\endedxtext



\endedxvertical


\beginedxvertical{Introduce Yourself}

\beginedxtext{Introduce Yourself}

Please make a post to introduce yourself in the appropriate discussion thread below.  Here are some questions that can help prompt you:

\begin{itemize}
\item
What is your name?
\item
Where are you from?  
\item
What is your educational background?  
\item
What do you do for a career (or what do you want to do)? 

\item
 Why did you want to take this course? 
 
\item
 Anything else you'd like us to know about yourself?  
\end{itemize}

(insert link below)

\endedxtext



\endedxvertical


\beginedxvertical{Accessibility}

\beginedxtext{Accessibility}

\mysection{Accessibility Statement}

Here at Rice and at EdX we strive to make our content as accessible as we can. The EdX platform adheres to the World Wide Web Consortium's \href{https://www.w3.org/TR/WCAG/}{Web Content Accessibility Guidelines (WCAG) 2.0}. If at any point you find that the content in this course is not up to standards, please contact Rice Online at riceonline@rice.edu.




\endedxtext



\endedxvertical
